% Document Type: LaTeX
% Master File: Egypto.tex
\documentclass[final]{article}
\usepackage{egypto}
\usepackage[psfonts]{hiero}

%%\overfullrule=5pt
\newcommand{\HieroTeX}{Hiero\TeX{}}
\title{\HieroTeX{}, A \LaTeX periment of hieroglyphic typesetting} \author{Serge
  Rosmorduc\\ rosmord@iut.univ-paris8.fr} \def\comment#1{}

\def\Backslash{$\backslash$}
\def\htimage#1{{#1\relax}}

\begin{document}
 \maketitle

 

\begin{center}
  \begin{hieroglyph}{\leavevmode \cartouche{\Cadrat{\CadratLineI{\Aca GO/65/}\CadratLine{\Aca GD/69/}\CadratLine{\Aca GD/52/}}\HinterSignsSpace
\loneSign{\ligAROBD}\HinterSignsSpace
\Cadrat{\CadratLineI{\Aca GE/55/}\CadratLine{\Aca GO/65/}}\HinterSignsSpace
\loneSign{\Aca GG/50/}\HinterSignsSpace
\Cadrat{\CadratLineI{\Aca GD/52/}\CadratLine{\Aca GD/79/}}\HinterSignsSpace
\Cadrat{\CadratLineI{\Aca GZ/40/}\CadratLine{\Aca GV/62/}}\HinterSignsSpace
\loneSign{\Aca GA/32/}}%
}\end{hieroglyph}
\end{center}
  
\section{License}
\begin{center}
  Sesh Nesout, a TeX package for hieroglyphic typesetting\\
  Copyright (C) 1993-2001 Serge Rosmorduc
\end{center}     
This program is free software; you can redistribute it and/or modify
it under the terms of the GNU General Public License as published by
the Free Software Foundation; either version 1, or (at your option)
any later version.
     
This program is distributed in the hope that it will be useful, but
WITHOUT ANY WA\-RRANTY; without even the implied warranty of
MERCHANTABI\-LITY or FITNESS FOR A PARTICULAR PURPOSE.  See the GNU
General Public License for more details.
     
You should have received a copy of the GNU General Public License
along with this program; if not, write to the Free Software
Foundation, Inc., 675 Mass Ave, Cambridge, MA 02139, USA.

\section{Introduction}
If you use \LaTeX{}, and want to typeset hieroglyphic texts, this
package is for you. If you want to typeset hieroglyphs, and don't know
\LaTeX, then have a look at the file {\tt EGypto.tex}, to get an idea
of what \LaTeX{} looks like. 

You can use also use the (ugly, but getting better) fonts for your own
programs. There are both a metafont and gsf (postscript type I) fonts.
The fonts were made with the GNU fontutils.

\section{Installation}
As \LaTeX{} works on so many machines, I can't really give you
detailed informations. So I assume you know a little about it.

\subsection{Windows (MikTeX)}

There are a number of \LaTeX{} versions available for windows. Most of
them are free. The installation procedure described here is for
Mik\TeX, which you can fetch at:
\begin{quotation}
  \texttt{http://www.miktex.org/2.1/index.html}
\end{quotation}

After installing MiKTex, you need to install the fonts and sty for
Hieroglyphs that you got from unpacking the tgz file. Note that Winzip
unpacks tgz files.

Put the MF and AUXMF folders in the following directories:
\begin{verbatim}
MiKTeX\fonts\source\Hierotex\MF
MiKTeX\fonts\source\Hierotex\AUXMF
\end{verbatim}

Put the sty files at the following directories:
\begin{verbatim}
MiKTeX\latex\hierotex\Hierltx.sty
MiKTeX\latex\hierotex\hiero.sty
\end{verbatim}

The next step is to run:
\begin{verbatim}
MiKTeX\miktex\config\config.bat
\end{verbatim}

Then make sure the sesh.exe file in in your path.

\subsection{Unix installation with tetex}

If you are using Unix and the tetex distribution, the simpler thing to
do is:
\begin{enumerate}
\item edit variable.mk in the HieroTeX directory.  You need to check
  the variables DESTDIR, BINDIR, TEXROOT are correct for you.
\item type ``\texttt{make tetex-install}''. You might need to be root for this
  to work. That's all.
\end{enumerate}

\paragraph{What to do if you can't log as root ?}
type:
\begin{verbatim}
mkdir $HOME/bin
mkdir $texmf
\end{verbatim}

edit variables.mk and set the following variables:
\begin{verbatim}
# Destination directory. Usually /usr, /usr/local or $(HOME)
DESTDIR=$(HOME)
# Place where the binary goes
BINDIR=$(DESTDIR)/bin
# Root of the tex tree the files should be installed into.
TEXROOT=$(DESTDIR)/texmf
# Tex Style file directory
TEXSTYLE=$(TEXROOT)/tex/latex/HieroTeX
# Font directory
FONTDIR=$(TEXROOT)/fonts/source/public/HieroTeX
\end{verbatim}

Now, get sure that the \texttt{texmf.cnf} configuration file for your system defines:
\begin{verbatim}
TEXMF = {$HOMETEXMF,!!$TEXMFMAIN}
\end{verbatim}
If this is not the case, you may define it as an environment variable.

Now, you can type \texttt{make tetex-install}

\subsection{Installing the postscript type 1 fonts on linux}

This is optionnal, but gives nice results for postscript and above all
pdf files. Beware : \textbf{to get a decent pdf output, you need to
  use pdflatex.} At least, that's what I have seen. Using
\texttt{dvips} and then \texttt{ps2pdf} produces a file where the
hieroglyphic fonts are still coded in type 3 postscript fonts, which
look horrible in acrobat. The version of ps2pdf shipped with the 3.0
debian actually produced an file acrobat couldn't read (well, at least
on my machine)

I'm afraid this is an area where things can be quite different from
one \TeX{} variant to another. There are two ways to go : user
install, and system-wide install. I will describe user-specific
install here. The system I use runs a Debian 3.0 Linux, with tetex
installed.

Download the fonts (file \texttt{HieroType1-VERSION.tgz}).  insert them in the
\texttt{texmf} hierarchy you used. For instance, I did put them in
\texttt{texmf/fonts/type1}, in my home directory (get sure the TEXMF
path is properly defined in texmf.cnf). You can simply run :
\begin{verbatim}
cd ; tar zxvf hieroType1.tgz
\end{verbatim}

Now, run \texttt{texhash}. The fonts should work with latex and dvips, and pdflatex.

The configuration files are in \texttt{texmf/dvips} for latex and
dvips, and in \texttt{texmf/pdftex} for \texttt{pdflatex}.  If things
work funny, have a look at the files \texttt{pdftex.cfg} and
\texttt{config.ps} in these directories, and compare them with your
system's default. If you want to make a system-wide install, the idea
is the same, except that I strongly suggest not to overwrite your
system's \texttt{config.ps} and \texttt{pdftex.cfg} with my own. The important lines to add are :
\begin{itemize}
\item in \texttt{config.ps} :
\begin{verbatim}
% This shows how to add your own map file.
% Remove the comment and adjust the name:
p +hierofonts.map
\end{verbatim}

\item in \texttt{pdftex.cfg} :
\begin{verbatim}
% This shows how to add your own map file. 
% Remove the comment and adjust the name:
map +hierofonts.map
map +diacrFonts.map
\end{verbatim}
\end{itemize}

\section{Typesetting hieroglyphs}

% I provide two ways of typesetting hieroglyphs. The first one only uses
% the power  of \LaTeX{}, and the second use a C program as a
% pre-processor.

% \subsection{The hard way}
% If you want to use only  \LaTeX, and nothing else, you must include
% the option \verb/hierLtx/ in your \verb=\documentstyle= line, like
% this: 
% \begin{verbatim}
% \documentstyle[hierLtx]{article}
% \end{verbatim}

%         To type a hieroglyphic text directly, you can use the 
% \verb.\.\verb.hieroglyphe. macro. You type the signs like this:
% \begin{center}
% \verb?\?\verb=hieroglyphe{A/1}= \\
% to get:\\  
% \begin {hieroglyph}              
%   A1
% \end {hieroglyph} % I cheat. too bad
% \end{center}
% Beware!  the numbers aren't exactly the one in Gardiner's grammar. To
% find which number gives you which sign, you may use the \verb/testfont/ 
% \TeX{} source, like this: 

% let's suppose you want the sign list for ``man''. You type (on a
% unix box, but the idea is the same elsewhere. Your input is {\sl slanted,} the computer's output is in typewriter font) :
% \begin{flushleft}
%   \tt {\sl tex testfont}\\
%   \verb#This is TeX--XeT, Version 3.1415--1.0 (C version 6.1)#\\
%   \verb#TeX--XeT Copyright (C) 1992 by Dante e.V.#\\
%   \verb#(/usr/TeX/lib/texmf/tex/plain/testfont.tex#\\
%   \verb#Name of the font to test =# {\sl A\/}\\
%   \verb#Now type a test command (\help for help):)#\\
%   \verb#*#{\sl \Backslash table  \vspace{2ex}\/}\\
%   \verb#*#{\sl \Backslash bye}\\
%   \verb#[1]#\\
%   \verb#Output written on testfont.dvi (1 page, 6088 bytes).#\\
%   \verb#Transcript written on testfont.log.#\\
% \end{flushleft}
% Then, you can print the table.

%         To type a complicated text, you can , like in the ``manuel'',
% type: `\verb=-=' to separate two groups horizontally, and `\verb=:='
% to separate two groups vertically. Accolades allow you to group signs.

% \begin{center}
%   \verb,\,\verb=hieroglyphe{{Q/3-X/1}:N/1}= \\
%   \mbox{\begin {hieroglyph}
%     p*t:pt
%   \end {hieroglyph}}
% \end{center} 
% nice, isn't it\footnote{Actually, I'm cheating. See the file
%   {\tt Ltxex.tex} for real example.}?



% \subsection{The ``Easy'' Way}

You need a C compiler, to build\footnote{see the file
  \texttt{BUILD.txt}} the \verb/sesh/ program, or a compiled version
thereof.  I intended the software to be called \EnPetit{\begin{hieroglyph}{\leavevmode \loneSign{\Aca GM/54/}\HinterSignsSpace
\Cadrat{\CadratLineI{\Aca GX/32/}\CadratLine{\Aca GN/66/}}\HwordSpace
\loneSign{\Aca GY/34/}\HinterSignsSpace
\loneSign{\Aca GA/32/}\HwordSpace
\Cadrat{\CadratLineI{\Aca GU/35/}\CadratLine{\Aca GD/69/}}\HinterSignsSpace
\loneSign{\Aca GY/40/}}\end{hieroglyph}}, but it's a little hard to type on a keyboard.

You might also want the {\em Inventaire des signes hi\'eroglyphiques
  en vue de leur saisie informatique, \/}~\cite{MANCOD} hereafter
called {\em manuel de codage\/} (in English, French, and German).

To use \texttt{sesh,} write your \LaTeX\ source with hieroglyphic texts in a
\verb|hieroglyph| environment. Then, use sesh as a filter to obtain a 
\LaTeX able result. As any polite program, sesh reads stdin and writes
on stdout. 

\paragraph{example:} 
if your \LaTeX{} file, called \verb/foo.htx/, is:
\begin{flushleft} \tt
  \verb/\/\verb/documentclass{article}/\\
  \verb/\/\verb/usepackage{hiero}/\\
  \verb.\.\verb/begin{document}/ \\
  \verb.\.\verb/begin{hieroglyph}/\\
  A1
  \verb/\/\verb/end{hieroglyph}/\\
  \verb/\/\verb/end{document}/\\
\end{flushleft}
you should proceed like this:
\begin{flushleft}
  \texttt{sesh} \texttt{<} \textsl{foo.htx} \texttt{>} \textsl{foo.tex}\\
  \texttt{latex} {\sl foo.tex}
\end{flushleft}
and you should get a dvi file containing the sign 
\begin{hieroglyph}{\leavevmode \loneSign{\Aca GA/32/}}\end{hieroglyph}.
Beware that you can't use \LaTeX{} comments between the 
beginning and the end of the \verb/hieroglyph/ environment. 


\subsubsection{The hiero package and its options}

To use \HieroTeX{}, you should include the \texttt{hiero} package,
like this :
\begin{verbatim}
\usepackage{hiero}
\end{verbatim}

If you want to use the hieroglyphic postscript fonts, you should specify the \texttt{psfonts} option :
\begin{verbatim}
\usepackage[psfonts]{hiero}
\end{verbatim}
In this case, you should have the postscript fonts installed (see
above). Please note that these fonts won't always look good in the dvi
output. You have to look at the final postscript or pdf output to get
an exact idea of the result. In particular, right-to-left signs won't
look correct in dvi.

\subsubsection{Hieroglyphs}

The ``alphabetic'' signs can be called like shown in
table~\ref{tab:alpha}.  The Letters A, i, a, w, b, p, f, m, n, r, h,
H, x, X, z, s, S, q, k, g, t, T, d, and D are used both for the
hieroglyphs and their transliteration.
\begin{table}[htbp]
  \begin{center}
    \leavevmode
    \EnPetit{
    \begin{tabular}{|c|c|c|} \hline
      \small\sc sign & \small\sc \vbox{\vskip 3pt\hbox{trans-}\hbox{litteration}} &
      \small\sc code \\ \hline
        \begin{hieroglyph}{\leavevmode \loneSign{\Aca GG/32/}}\end{hieroglyph} & \eg A & A\\
        \begin{hieroglyph}{\leavevmode \loneSign{\Aca GM/48/}}\end{hieroglyph} & \eg i & i\\
        \begin{hieroglyph}{\leavevmode \loneSign{\Aca GD/69/}}\end{hieroglyph} & \eg a & a\\
        \begin{hieroglyph}{\leavevmode \loneSign{\Aca GG/77/}}\end{hieroglyph} & \eg w & w\\
        \begin{hieroglyph}{\leavevmode \loneSign{\Aca GZ/40/}}\end{hieroglyph} & \eg w & W\\
        \begin{hieroglyph}{\leavevmode \loneSign{\Aca GD/92/}}\end{hieroglyph} & \eg b & b\\
        \begin{hieroglyph}{\leavevmode \loneSign{\Aca GQ/34/}}\end{hieroglyph} & \eg p & p\\
        \begin{hieroglyph}{\leavevmode \loneSign{\Aca GI/41/}}\end{hieroglyph} & \eg f & f\\
        \begin{hieroglyph}{\leavevmode \loneSign{\Aca GG/50/}}\end{hieroglyph} & \eg m & m\\
        \begin{hieroglyph}{\leavevmode \loneSign{\Aca GN/66/}}\end{hieroglyph} & \eg n & n\\
        \begin{hieroglyph}{\leavevmode \loneSign{\Aca GS/34/}}\end{hieroglyph} & \eg n & N\\
        \begin{hieroglyph}{\leavevmode \loneSign{\Aca GD/52/}}\end{hieroglyph} & \eg r & r\\
        \begin{hieroglyph}{\leavevmode \loneSign{\Aca GO/35/}}\end{hieroglyph} & \eg h & h\\
        \begin{hieroglyph}{\leavevmode \loneSign{\Aca GV/59/}}\end{hieroglyph} & \eg H & H\\
        \begin{hieroglyph}{\leavevmode \loneSign{\Aca GAa/32/}}\end{hieroglyph} & \eg x & x\\
        \begin{hieroglyph}{\leavevmode \loneSign{\Aca GF/63/}}\end{hieroglyph} & \eg X & X\\
        \begin{hieroglyph}{\leavevmode \loneSign{\Aca GO/65/}}\end{hieroglyph} & \eg z & z\\
        \begin{hieroglyph}{\leavevmode \loneSign{\Aca GS/63/}}\end{hieroglyph} & \eg s & s\\
        \begin{hieroglyph}{\leavevmode \loneSign{\Aca GN/69/}}\end{hieroglyph} & \eg S & S\\
        \begin{hieroglyph}{\leavevmode \loneSign{\Aca GN/60/}}\end{hieroglyph} & \eg q & q\\
        \begin{hieroglyph}{\leavevmode \loneSign{\Aca GV/62/}}\end{hieroglyph} & \eg k & k\\
        \begin{hieroglyph}{\leavevmode \loneSign{\Aca GW/44/}}\end{hieroglyph} & \eg g & g\\
        \begin{hieroglyph}{\leavevmode \loneSign{\Aca GX/32/}}\end{hieroglyph} & \eg t & t\\
        \begin{hieroglyph}{\leavevmode \loneSign{\Aca GV/44/}}\end{hieroglyph} & \eg T & T\\
        \begin{hieroglyph}{\leavevmode \loneSign{\Aca GD/79/}}\end{hieroglyph} & \eg d & d\\
        \begin{hieroglyph}{\leavevmode \loneSign{\ligAROBD}}\end{hieroglyph} & \eg D & D\\ \hline
      \end{tabular}}
    \end{center}
    \caption{Alphabetical signs\label{tab:alpha}}
  \end{table}
  \sloppy Signs are named, either after their transliterations (for
  example, \verb|xpr| is \EnPetit{\begin{hieroglyph}{\leavevmode \loneSign{\Aca GL/32/}}\end{hieroglyph}}%
) \fussy or after their Gardiner Code. See appendix~\ref{sec:signes}
for a list of codes. For a complete list, see~\cite{MANCOD}, which
intends to be {\em the\/} format for encoding hieroglyphic texts.

A few codes are used for signs variants; these are \verb/pA'/ and
\verb/k'/ which write
\begin{hieroglyph}{\leavevmode \loneSign{\Aca GG/75/}}\end{hieroglyph} and 
\begin{hieroglyph}{\leavevmode \loneSign{\Aca GV/63/}}\end{hieroglyph}.


\subsubsection{Special signs}
You won't type only hieroglyphs. So: 
\begin{description}
\item[..] is a space
\item[.] is a quater-space
\item[//] is 
  \begin{hieroglyph}{\leavevmode \hachureg }\end{hieroglyph}

\item[h/] is 
  \begin{hieroglyph}{\leavevmode \hachureh }\end{hieroglyph}

\item[v/] is 
  \begin{hieroglyph}{\leavevmode \hachurev }\end{hieroglyph}

\item[/] is 
  \begin{hieroglyph}{\leavevmode \hachuret }\end{hieroglyph}


\item[o] is 
  \begin{hieroglyph}{\leavevmode \pointrouge}\end{hieroglyph}

\item[O] is 
  \begin{hieroglyph}{\leavevmode \pointnoir}\end{hieroglyph}

\end{description}

\subsubsection{Grouping Signs}

You can type adjacent signs with `-' , stack signs with `:', group
signs with parenthesis, and use `\verb.*.' to separate signs on the
same level. 

A few examples: 
\begin{flushleft}
  \verb.\.\verb/begin{hieroglyph}/\\
  \verb/p*t:pt-..-A-..-n:(x:t)*U30-A-xAst-qmA/\\
  \verb/./\verb/end{hieroglyph}/\\
\end{flushleft}
gives 
\begin{hieroglyph}{\leavevmode \Cadrat{\CadratLineI{\Aca GQ/34/\hfill\Aca GX/32/}\CadratLine{\Aca GN/32/}}\HinterSignsSpace
\HfullSpace \HinterSignsSpace
\loneSign{\Aca GG/32/}\HinterSignsSpace
\HfullSpace \HinterSignsSpace
\Cadrat{\CadratLineI{\Aca GN/66/}\CadratLine{\InternalCadrat{\CadratLineI{\Aca GAa/32/}\CadratLine{\Aca GX/32/}}\hfill\Aca GU/62/}}\HinterSignsSpace
\loneSign{\Aca GG/32/}\HinterSignsSpace
\loneSign{\Aca GN/56/}\HinterSignsSpace
\loneSign{\Aca GT/49/}}\end{hieroglyph}

you can type a `=' sign {\em after\/}  a `-' or a `:' to indicate a
grammatical ending. More important, a {\em space, or any number of
  spaces, tabulations, or newlines,}  indicate a word
ending\footnote{Note that in the ``manuel'', a word ending is {\em
    one\/} space. This is extremely inconvenient for \LaTeX{}, and, in
  fact, for the general purpose of directly typing the text.}
I strongly recommend that you type the word-separation spaces. The
package will then try to cut the lines at word-separations, which is
nicer and easier to read\footnote{You can use spaces directly to
  separate words, if you want, without using a `-'. but this won't be
  conform to the manuel encoding.}.


\subsubsection{Sign Modifiers}

\begin{description}
\item[\Backslash{}\Backslash{}] after a signs inverts it:
  \verb|A1-A1\| writes 
  \begin{hieroglyph}{\leavevmode \loneSign{\Aca GA/32/}\HinterSignsSpace
\loneSign{\Hrevert{\Aca GA/32/}}}\end{hieroglyph}

\item[\Backslash{}s$n$] where $n$ is 1,2,3 or 4, diminishes the 
  sign size: 
  \begin{center}
\verb/A1-A1\s1-A1\s2-A1\s3-A1\s4/ 
  \end{center}
draws
  \begin{hieroglyph}{\leavevmode \loneSign{\Aca GA/32/}\HinterSignsSpace
\loneSign{{\Hsmaller\Aca GA/32/}}\HinterSignsSpace
\loneSign{{\Hsmaller\Hsmaller\Aca GA/32/}}\HinterSignsSpace
\loneSign{{\Hsmaller\Hsmaller\Hsmaller\Aca GA/32/}}\HinterSignsSpace
\loneSign{{\Hsmaller\Hsmaller\Hsmaller\Hsmaller\Aca GA/32/}}}\end{hieroglyph}
  
\item[\Backslash{}R$n$] rotates a figure by the angle $n$. This
  excedes normal dvi capacities, so you must use a suitable graphic
  driver : \texttt{dvips}  for postscript, or \texttt{pdftex} for pdf.
To do this, include:
\begin{verbatim}
\usepackage[dvips]{graphicx}
\end{verbatim}
or 
\begin{verbatim}
\usepackage[pdftex]{graphicx}
\end{verbatim}
in your headings.
  \verb|anx\R30|-G5
  \begin{hieroglyph}{\leavevmode \loneSign{\Hrotate{-30}{\Aca GS/68/}}\HinterSignsSpace
\loneSign{\Aca GG/36/}}\end{hieroglyph}
\end{description}
Alternatively, you can use the option \verb|useGraphicx| when loading
the hiero package :
\begin{verbatim}
\package[useGraphicx]{hiero}
\end{verbatim}
Note that this option also switches on the use of postscript type1
hieroglyphic fonts.

\subsubsection{Cartouches and parenthesis}

\begin{description}
\item[\tt <- and ->] draw a cartouche around the embedded text:
\begin{verbatim}
 <-ra-mn:n-xpr->
\end{verbatim}
  \begin{hieroglyph}{\leavevmode \cartouche{\loneSign{\Aca GN/36/}\HinterSignsSpace
\Cadrat{\CadratLineI{\Aca GY/36/}\CadratLine{\Aca GN/66/}}\HinterSignsSpace
\loneSign{\Aca GL/32/}}%
}\end{hieroglyph}

  
\item[\tt <S and >] draw a serekh: 
\begin{verbatim}
G5 -<S-nTr-X:t->
\end{verbatim}
  \begin{hieroglyph}{\leavevmode \loneSign{\Aca GG/36/}\HwordSpace
\serekh{\loneSign{\Aca GR/39/}\HinterSignsSpace
\Cadrat{\CadratLineI{\Aca GF/63/}\CadratLine{\Aca GX/32/}}}%
}\end{hieroglyph}
  

\item[\tt <H >] draw a \htimage{\eg Hwt-}sign.
\begin{verbatim}
<H-pA-i-i-A1-Htp:t*p>
\end{verbatim}
  \begin{hieroglyph}{\leavevmode \chateau{\loneSign{\Aca GG/74/}\HinterSignsSpace
\loneSign{\Aca GM/48/}\HinterSignsSpace
\loneSign{\Aca GM/48/}\HinterSignsSpace
\loneSign{\Aca GA/32/}\HinterSignsSpace
\Cadrat{\CadratLineI{\Aca GR/35/}\CadratLine{\Aca GX/32/\hfill\Aca GQ/34/}}}%
}\end{hieroglyph}
\end{description}
The letters {\em b, m and e,} used after the beginning sign, allow you
to draw only the beginning, middle, or ending of construction. 
So, for example, 
\verb/<Se-nTr->/ writes
\begin{hieroglyph}{\leavevmode \endOfserekh{\loneSign{\Aca GR/39/}}%
}\end{hieroglyph}.

The parenthesis are:
\begin{description}
\item[{[[...]]}] for 
  \begin{hieroglyph}{\leavevmode \ediefface{{\rm  text }}%
}\end{hieroglyph};
\item[{[\{...\}]}] for 
  \begin{hieroglyph}{\leavevmode \edidisparu{{\rm  text }}%
}\end{hieroglyph};
\item[{[\&...\&]}] for 
  \begin{hieroglyph}{\leavevmode \ediajoutauteur{{\rm  text }}%
}\end{hieroglyph}.
\end{description}


\subsubsection{Shading and stacking}

\verb|-#-| \ldots \verb|-#-| shades a text\footnote{The previous
  versions of sesh allowed one to do without the `-'. but I decided to
  revert to the manual, to be able to use the \# to stack signs.}:
\verb/-#-A1-A2-#-/ writes 
\begin{hieroglyph}{\leavevmode \hachure{\loneSign{\Aca GA/32/}\HinterSignsSpace
\loneSign{\Aca GA/33/}}%
}\end{hieroglyph}.

You can shade a part of a group with \verb.#//., 
\verb.#h/., \verb.#v/., and \verb.#/.:
\verb.A1#//. is 
\begin{hieroglyph}{\leavevmode \hachurega{\loneSign{\Aca GA/32/}}}\end{hieroglyph}; 
\verb.A1#h/. is 
\begin{hieroglyph}{\leavevmode \hachureha{\loneSign{\Aca GA/32/}}}\end{hieroglyph}; 
\verb.A1#v/. is 
\begin{hieroglyph}{\leavevmode \hachureva{\loneSign{\Aca GA/32/}}}\end{hieroglyph}, and 
\verb.A1#/. is 
\begin{hieroglyph}{\leavevmode \hachureta{\loneSign{\Aca GA/32/}}}\end{hieroglyph}.

It is possible to specify which quarters of a cadrat will be shaded.
This is done by writing \verb|#| after the cadrat, and the numbers of
the cadrat quarters to shade. The quarters are numbered this way:
\begin{center}
  \begin{tabular}{cc}
    1 & 2 \\
    3 & 4\\
  \end{tabular}
\end{center}
The following example demonstrates the system:
\begin{verbatim}
  n:(x:t)*U30#13-..-n:(x:t)*U30#124-..-n:(x:t)*U30#14-..-n:(x:t)*U30#1234
\end{verbatim}
\begin{hieroglyph}{\leavevmode \newShading{XOXO}{\Cadrat{\CadratLineI{\Aca GN/66/}\CadratLine{\InternalCadrat{\CadratLineI{\Aca GAa/32/}\CadratLine{\Aca GX/32/}}\hfill\Aca GU/62/}}}\HinterSignsSpace
\HfullSpace \HinterSignsSpace
\newShading{XXOX}{\Cadrat{\CadratLineI{\Aca GN/66/}\CadratLine{\InternalCadrat{\CadratLineI{\Aca GAa/32/}\CadratLine{\Aca GX/32/}}\hfill\Aca GU/62/}}}\HinterSignsSpace
\HfullSpace \HinterSignsSpace
\newShading{XOOX}{\Cadrat{\CadratLineI{\Aca GN/66/}\CadratLine{\InternalCadrat{\CadratLineI{\Aca GAa/32/}\CadratLine{\Aca GX/32/}}\hfill\Aca GU/62/}}}\HinterSignsSpace
\HfullSpace \HinterSignsSpace
\newShading{XXXX}{\Cadrat{\CadratLineI{\Aca GN/66/}\CadratLine{\InternalCadrat{\CadratLineI{\Aca GAa/32/}\CadratLine{\Aca GX/32/}}\hfill\Aca GU/62/}}}\HwordSpace
\newShading{XOXO}{\Cadrat{\CadratLineI{\Aca GN/66/}\CadratLine{\InternalCadrat{\CadratLineI{\Aca GAa/32/}\CadratLine{\Aca GX/32/}}\hfill\Aca GU/62/}}}\HinterSignsSpace
\newShading{XXOX}{\Cadrat{\CadratLineI{\Aca GN/66/}\CadratLine{\InternalCadrat{\CadratLineI{\Aca GAa/32/}\CadratLine{\Aca GX/32/}}\hfill\Aca GU/62/}}}\HinterSignsSpace
\newShading{XOOX}{\Cadrat{\CadratLineI{\Aca GN/66/}\CadratLine{\InternalCadrat{\CadratLineI{\Aca GAa/32/}\CadratLine{\Aca GX/32/}}\hfill\Aca GU/62/}}}\HinterSignsSpace
\newShading{XXXX}{\Cadrat{\CadratLineI{\Aca GN/66/}\CadratLine{\InternalCadrat{\CadratLineI{\Aca GAa/32/}\CadratLine{\Aca GX/32/}}\hfill\Aca GU/62/}}}}\end{hieroglyph}

\comment{
There is a limited possibility for shading other parts of groups.
For example: \verb.p*t#v/:pt.. 
\begin{hieroglyph}{\leavevmode \Cadrat{\CadratLineI{\Aca GQ/34/\hfill\hachureva{\Aca GX/32/}}\CadratLine{\Aca GN/32/}}}\end{hieroglyph}, and a new shading system will probably appear in the
next ``manuel''.  
}

\verb/#/ on his own allows to stack signs: 
\verb/m#a/ is 
\begin{hieroglyph}{\leavevmode \hsuperpose{\loneSign{\Aca GG/50/}}{\loneSign{\Aca GD/69/}}}\end{hieroglyph}.

\subsubsection{Text Orientation}

If you use a version of TeX--Xet (a patch designed to allow you to
type right-to-left text as well as left-to-right, with the benefit of
the line-cutting system), It is possible to reverse a whole part of
the text. Simply type \verb/+dg/ right-to-left, and \verb/+gd/ for
left-to-right (this is {\em not\/} part of the {\em manuel\/}).

\expandafter\ifx\csname beginL\endcsname\relax % be sure the file 
\def\beginL{\relax}             %can be compiled anyway.
\def\beginR{\relax} \def\endR{\relax} \def\endL{\relax}

{\bf BEWARE !!!! this file was not compiled with TeX--Xet, so the
  example bellow is false. You should have gotten a dvi version of
  this file (called \verb/doc.dvi/), without this message.} \fi

Example:
\begin{hieroglyph}{\leavevmode \rightleft\HwordSpace
\loneSign{\Aca GG/38/}\HwordSpace
\Cadrat{\CadratLineI{\Aca GE/32/}\CadratLine{\Aca GD/73/}}\HwordSpace
\Cadrat{\CadratLineI{\Aca GN/59/}\CadratLine{\Aca GD/69/}}\HwordSpace
\loneSign{\Aca GG/50/}\HwordSpace
\loneSign{\Aca GS/75/}\HinterSignsSpace
\Cadrat{\CadratLineI{\Aca GX/32/}\CadratLine{\Aca GO/80/}}\HwordSpace
\loneSign{\Aca GG/49/}\HwordSpace
\leftright\HwordSpace
\loneSign{\Aca GS/68/}\HwordSpace
\loneSign{\Aca GG/38/}\HwordSpace
\Cadrat{\CadratLineI{\Aca GE/32/}\CadratLine{\Aca GD/73/}}\HwordSpace
\Cadrat{\CadratLineI{\Aca GN/59/}\CadratLine{\Aca GD/69/}}\HwordSpace
\loneSign{\Aca GG/50/}\HwordSpace
\loneSign{\Aca GS/75/}\HinterSignsSpace
\Cadrat{\CadratLineI{\Aca GX/32/}\CadratLine{\Aca GO/80/}}\HwordSpace
\loneSign{\Aca GG/49/}}\end{hieroglyph}, made like that: 
\begin{verbatim}
  +dg G7 E1:D40 xa:a m wAs-t:niwt nbty +gd anx G7 E1:D40 xa:a m
  wAs-t:niwt nbty
\end{verbatim}

Testing compatibility with `\textbackslash'
\begin{hieroglyph}{\leavevmode \rightleft\HwordSpace
\loneSign{\Hrevert{\Aca GA/35/}}\HinterSignsSpace
\loneSign{\Aca GC/33/}\HinterSignsSpace
\Cadrat{\CadratLineI{\Aca GAa/32/\hfill\Aca GX/32/}\CadratLine{\Aca GI/41/}}\HinterSignsSpace
\loneSign{\Aca GG/77/}\HinterSignsSpace
\Cadrat{\CadratLineI{\Aca GD/92/}\CadratLine{\Aca GN/66/}}\HinterSignsSpace
\Cadrat{\CadratLineI{\Aca GN/39/}\CadratLine{\Aca GI/41/}}\HinterSignsSpace
\loneSign{\Aca GG/50/}\HinterSignsSpace
\Cadrat{\CadratLineI{\Aca GN/58/}\CadratLine{\Aca GX/32/\hfill\Aca GZ/32/}}\HinterSignsSpace
\loneSign{\Aca GR/46/}\HinterSignsSpace
\Cadrat{\CadratLineI{\Aca GX/32/}\CadratLine{\Aca GX/32/}}\HwordSpace
\leftright\HwordSpace
\loneSign{\Hrevert{\Aca GA/35/}}\HinterSignsSpace
\loneSign{\Aca GC/33/}\HinterSignsSpace
\Cadrat{\CadratLineI{\Aca GAa/32/\hfill\Aca GX/32/}\CadratLine{\Aca GI/41/}}\HinterSignsSpace
\loneSign{\Aca GG/77/}\HinterSignsSpace
\Cadrat{\CadratLineI{\Aca GD/92/}\CadratLine{\Aca GN/66/}}\HinterSignsSpace
\Cadrat{\CadratLineI{\Aca GN/39/}\CadratLine{\Aca GI/41/}}\HinterSignsSpace
\loneSign{\Aca GG/50/}\HinterSignsSpace
\Cadrat{\CadratLineI{\Aca GN/58/}\CadratLine{\Aca GX/32/\hfill\Aca GZ/32/}}\HinterSignsSpace
\loneSign{\Aca GR/46/}\HinterSignsSpace
\Cadrat{\CadratLineI{\Aca GX/32/}\CadratLine{\Aca GX/32/}}}\end{hieroglyph}, made like that: 
\begin{verbatim}
  +dg A4\-C2-x*t:f-w-b:n-N8:f-m-Axt:t*1-iAb-t:t -+gd
  A4\-C2-x*t:f-w-b:n-N8:f-m-Axt:t*1-iAb-t:t
\end{verbatim}

There is a more general right-to-left builtin, but it won't allow
line-cutting: `+rl' and `+lr'. 
example:
\begin{verbatim}
 +rl G5 -nTr-ms-w-t:Z2 +lr anx G5 -nTr-ms-w-t:Z2 
\end{verbatim}
\begin{hieroglyph}{\leavevmode \boxrightleft{\Cadrat{\CadratLineI{\Aca GX/32/}\CadratLine{\Aca GZ/33/}}\HinterSignsSpace
\loneSign{\Aca GG/77/}\HinterSignsSpace
\loneSign{\Aca GF/62/}\HinterSignsSpace
\loneSign{\Aca GR/39/}\HwordSpace
\loneSign{\Aca GG/36/}}%
\HwordSpace
\loneSign{\Aca GS/68/}\HwordSpace
\loneSign{\Aca GG/36/}\HwordSpace
\loneSign{\Aca GR/39/}\HinterSignsSpace
\loneSign{\Aca GF/62/}\HinterSignsSpace
\loneSign{\Aca GG/77/}\HinterSignsSpace
\Cadrat{\CadratLineI{\Aca GX/32/}\CadratLine{\Aca GZ/33/}}}\end{hieroglyph}. It will work in any case.

Finally, if your ultimate output is postscript, you can use packages
like pstricks to invert left-to-right text and make it into
right-to-left.

For Column-writing, the \LaTeX{} macro \verb/\EnColonne/ can be used.
Its first argument is the column width. The one given in the following
example isn't bad. 
\begin{flushleft}
  \Backslash{}\verb/EnColonne[1.2\Htm]{/\\
  \Backslash\verb/begin{hieroglyph}/\\
  \verb/#def wAt N31#/
  \verb/sw*(t:di) -Htp -wp-wAt-wAt-wAt-!/\\
  \verb/E16-nb-tA:idb\s2*Z1-Dsr-r-xAst-!/\\
  \Backslash\verb/end{hieroglyph}/\\
  \verb/}/\\
\end{flushleft}
\begin{center}
\EnColonne[1.2\Htm]{
  \begin{hieroglyph}{\leavevmode \LoneHorizontalLine{\loneSign{\Aca GM/54/}\HinterSignsSpace
\Cadrat{\CadratLineI{\Aca GX/32/}\CadratLine{\Aca GX/40/}}}\HinterSignsSpace
\loneSign{\Aca GR/35/}\HwordSpace
\loneSign{\Aca GF/44/}\HinterSignsSpace
\loneSign{\Aca GN/62/}\HinterSignsSpace
\loneSign{\Aca GN/62/}\HinterSignsSpace
\loneSign{\Aca GN/62/}}\end{hieroglyph}\HendOfLine
\begin{hieroglyph}{\leavevmode
\loneSign{\Aca GE/48/}\HinterSignsSpace
\loneSign{\Aca GV/61/}\HinterSignsSpace
\loneSign{\Aca GN/47/}\HinterSignsSpace
\LoneHorizontalLine{\loneSign{{\Hsmaller\Hsmaller\Aca GN/51/}}\HinterSignsSpace
\loneSign{\Aca GZ/32/}}\HinterSignsSpace
\loneSign{\Aca GD/78/}\HinterSignsSpace
\loneSign{\Aca GD/52/}\HinterSignsSpace
\loneSign{\Aca GN/56/}}\end{hieroglyph}}
\end{center}
It is often necessary to use the \verb/\s/ construct to tune the sizes
of the signs in column writing. 


\subsubsection{Some text}

\verb|+l .. +s| allows inserting of \LaTeX{} text;  
\verb/|....-/ allows the insertion of superscript text:
\begin{verbatim}
<-+l Ramses II +s-> -A1-|1,2-di-anx
\end{verbatim}
\begin{hieroglyph}{\leavevmode \cartouche{{\rm  Ramses II }}%
\HwordSpace
\loneSign{\Aca GA/32/}\HinterSignsSpace
\traittexte{1,2}\HinterSignsSpace
\loneSign{\Aca GX/40/}\HinterSignsSpace
\loneSign{\Aca GS/68/}}\end{hieroglyph}


\subsubsection{kerning}

A non-manual feature of the present system allows to kern the
signs. the sequence \verb.\!. is a negative space that can be used
both in horizontal context:
\begin{verbatim}
zA-\!\!ra:.
\end{verbatim}
\mbox{
  \begin{hieroglyph}{\leavevmode \loneSign{\Aca GG/73/}\HinterSignsSpace
\negAROBspace\negAROBspace\Cadrat{\CadratLineI{\Aca GN/36/}\CadratLine{\HquarterSpace }}}\end{hieroglyph}}
and in vertical one: 
\begin{verbatim}
nD-D:\!\!\!\!Hr\s1
\end{verbatim}
  \mbox{
    \begin{hieroglyph}{\leavevmode \loneSign{\Aca GAa/55/}\HinterSignsSpace
\Cadrat{\CadratLineI{\ligAROBD}\negAROBvspace\negAROBvspace\negAROBvspace\negAROBvspace\CadratLine{{\Hsmaller\Aca GD/33/}}}}\end{hieroglyph}}


\subsubsection{Size}

The size of signs is controled by various \LaTeX{} macros:
\begin{itemize}
\item \verb/\DisplayHieroglyphs/ write the hieroglyphic texts in its
  scope in a  big size, with a large spacing between lines so that the
  reading is easy. 
\item \verb/\TextHieroglyphs/ does the contrary. It is better for
  in-text text. 
\item \verb/\EnPetit{}/  writes its argument in the same format as 
  \verb/\TextHieroglyphs/ does. 
\item \verb/\SmallerText/ lowers the size of the signs. 
\end{itemize}

{\verb/\DisplayHieroglyphs/ \DisplayHieroglyphs
This is an example: 
\begin{center}
  \mbox{\begin{hieroglyph}{\leavevmode \loneSign{\Aca GD/35/}\HinterSignsSpace
\loneSign{\Aca GN/66/}\HinterSignsSpace
\Cadrat{\CadratLineI{\Aca GD/35/}\CadratLine{\Aca GD/52/}}\HwordSpace
\loneSign{\Aca GD/52/}\HwordSpace
\Cadrat{\CadratLineI{\Aca GD/52/}\CadratLine{\Aca GX/32/}\CadratLine{\Aca GD/69/}}\HwordSpace
\Cadrat{\CadratLineI{\Aca GD/35/}\CadratLine{\Aca GI/41/}}}\end{hieroglyph}}
\end{center}
looks good.  \verb/\TextHieroglyphs/
\TextHieroglyphs \begin{hieroglyph}{\leavevmode \loneSign{\Aca GD/35/}\HinterSignsSpace
\loneSign{\Aca GN/66/}\HinterSignsSpace
\Cadrat{\CadratLineI{\Aca GD/35/}\CadratLine{\Aca GD/52/}}\HwordSpace
\loneSign{\Aca GD/52/}\HwordSpace
\Cadrat{\CadratLineI{\Aca GD/52/}\CadratLine{\Aca GX/32/}\CadratLine{\Aca GD/69/}}\HwordSpace
\Cadrat{\CadratLineI{\Aca GD/35/}\CadratLine{\Aca GI/41/}}}\end{hieroglyph} is  better if you are in the bulk of the text.
And  \verb/\SmallerText/ \SmallerText
 \begin{hieroglyph}{\leavevmode \loneSign{\Aca GD/35/}\HinterSignsSpace
\loneSign{\Aca GN/66/}\HinterSignsSpace
\Cadrat{\CadratLineI{\Aca GD/35/}\CadratLine{\Aca GD/52/}}\HwordSpace
\loneSign{\Aca GD/52/}\HwordSpace
\Cadrat{\CadratLineI{\Aca GD/52/}\CadratLine{\Aca GX/32/}\CadratLine{\Aca GD/69/}}\HwordSpace
\Cadrat{\CadratLineI{\Aca GD/35/}\CadratLine{\Aca GI/41/}}}\end{hieroglyph} is smaller
and \verb/\SmallerText/ \SmallerText
 \begin{hieroglyph}{\leavevmode \loneSign{\Aca GD/35/}\HinterSignsSpace
\loneSign{\Aca GN/66/}\HinterSignsSpace
\Cadrat{\CadratLineI{\Aca GD/35/}\CadratLine{\Aca GD/52/}}\HwordSpace
\loneSign{\Aca GD/52/}\HwordSpace
\Cadrat{\CadratLineI{\Aca GD/52/}\CadratLine{\Aca GX/32/}\CadratLine{\Aca GD/69/}}\HwordSpace
\Cadrat{\CadratLineI{\Aca GD/35/}\CadratLine{\Aca GI/41/}}}\end{hieroglyph} is even smaller
}

\subsubsection{Macros}
There is a limited macro facility (again, not part of the {\em manuel\/}). To
define a macro, type
\begin{verbatim}
#def MACRONAME Body#
\end{verbatim}
Note the second `\verb/#/' sign.  MACRONAME can be any unused sequence
of letters, simple quote, and point.

This can be useful if a name appears often. 

Example:

\begin{minipage}{15em}
\tt
\noindent\verb|\begin{|hierogl{}yph\verb|}| % to avoid interpretation !!!!!
\vspace{-2ex}
\begin{verbatim}
#def 'tA tA:N23*Z1#
tr-t:r-n-wn:n:n-k-tp-Z1-'tA
\end{hieroglyph}
\end{verbatim}
\end{minipage}
\vspace{1ex}

\begin{hieroglyph}{\leavevmode \loneSign{\Aca GM/37/}\HinterSignsSpace
\Cadrat{\CadratLineI{\Aca GX/32/}\CadratLine{\Aca GD/52/}}\HinterSignsSpace
\loneSign{\Aca GN/66/}\HinterSignsSpace
\Cadrat{\CadratLineI{\Aca GE/66/}\CadratLine{\Aca GN/66/}\CadratLine{\Aca GN/66/}}\HinterSignsSpace
\loneSign{\Aca GV/62/}\HinterSignsSpace
\loneSign{\Aca GD/32/}\HinterSignsSpace
\loneSign{\Aca GZ/32/}\HinterSignsSpace
\Cadrat{\CadratLineI{\Aca GN/47/}\CadratLine{\Aca GN/54/\hfill\Aca GZ/32/}}}\end{hieroglyph}
\vspace{1ex}

{\sloppy{} Note that, once defined, these macros can be used in all
  subsequent \verb|hieroglyph| environment.  }
\fussy

A few useful definitions: 
\begin{verbatim}
#def Hm' N42#
#def bdS A7#
#def bin G37#
#def arm D40#
#def knife T30#
#def copper N34#
#def di' D37#
#def hrw h-r:W*ra-Z1#
#def ink nw:k':A1#
#def inr O39#
#def iw' D54#
#def k' V31A#
#def nn M22-M22-n:n#
#def nxt A24#
#def rmn D41#
#def king A42#
#def sxA A2#
#def wnn wn:n:n#
\end{verbatim}

\subsection{\LaTeX\ macros}

If you want the small signs to be on the baseline, type
\verb|\SurLigne| out of a hieroglyph environment; the default is to
center signs. To switch back to centered signs, type \verb|\Centrer|.
In a cartouche, the text is always centered. You can handle the
line-cutting algorithm by changing the two \TeX\ macros
\verb|\Hrp| and \verb|\Hitmts|; the first rules what is
inserted between signs, the second between words.  The default values
tend to protect you from cutting a word, at the cost of some white
space.



\subsection{New Sign Definition}

It is now possible to define new signs for \HieroTeX{}. The difficult
part is that you have to create a font which \TeX{} can use, which is
a bit \TeX{}nical. The font editor for tksesh can be used for such
tasks. Another option is to create postscript type 1 font, which can be used by \TeX{}.

Once the font is created and installed in TeX, you must declare both
the font and the character to use it.

\subsubsection{Character Definitions}


First, you must create a file which describes the codes for the
characters in the font. The file format is rather simple:
on each line, you should have:
\begin{quotation}
  \textsl{SIGNCODE SPACE TEX\_NAME\_FOR\_THE\_FONT SPACE SIGN\_NUMBER\_IN\_FONT} 
\end{quotation}
at the time being, note that the fonts for \HieroTeX{} contain both
left-to-right and right-to-left sign (it would be possible to reverse
the signs with postscript, but it wouldn't appear correctly in dvi
files). So, restrict yourself to the first 128 character codes. For
reversal to work, character $128 + n$ should be character $n$
reversed.

The file which contains new character definitions may be:
\begin{itemize}
\item /usr/local/lib/sesh/fontDef.txt
\item \$HOME/fontDef.txt
\item specified when starting sesh, with the -def option:
\begin{verbatim}
sesh -def mydefs.txt <toto.htx >toto.tex
\end{verbatim}
\end{itemize}
\subsubsection{Font Declaration}

Second, the font must be declared in your \LaTeX{} file:
\begin{verbatim}
\declareHieroGlyphicFont{HIEROTEXFONTNAME}{TEXFONTNAME}
\end{verbatim}

\subsubsection{Example}

The files \texttt{testNewSigns.tex, titi.mf and titi\_font.mf} contain
a complete example. The \texttt{bzr} font is not needed, but it might
be edited using \texttt{fontedit.tcl,} a program which is available
with tksesh. The latest version of \texttt{fontedit.tcl} is able to
create metafont (and postscript type 1) sources. Note that postscript
type 1 fonts are not very easy to integrate in a latex environment.

\section{Typesetting transliterations and references}
The commands here described are defined in the \verb=egypto.sty= file.

\subsection{Settings}

\begin{description}
  \item[\Backslash Montitre\{...\}] allows you to define the document's title 
 (To be used in Cross-references )

 \item[\Backslash eg] to use the transliteration font

 \item[\Backslash def\Backslash SourceTexte\{Name of the text\}] to tell LaTeX what
text you are  typing

\item[\Backslash def\Backslash EXEMPLE\{example\}] To use this style in English

\end{description}

\subsection{Transliteration}

 There are currently three ways to write some transliteration
 (apart from \verb/\eg/)

\begin{enumerate}
 \item the translit environment:
 it takes 3 arguments: 
\begin{itemize}
        \item name of the text
        \item page number or recto/verso (exxs \verb/{4}/ or \verb/{recto}/
        \item line number(s) (could be column number) (exxs.
\verb/{4}/ \verb/{4-9}/ 
        \item an alternative (possibly empty) reference in free text.
\end{itemize}

 One important thing is that TeX ``understands'' these numbers.
 i.e.\ there are commands to change the line number and the page 
 number. such things are useful for cross-references

 you can  type a transliteration, and use ``\verb/\traduction/'' to start
 typing the translation.

 Example:

\begin{verbatim}
\begin{translit}{O. foobar XIV}{verso}{10-15}{LES 12,2-5}
  iw.i rx.kw mdw nTr
\traduction
  I know the hieroglyphs
\end{translit}
\end{verbatim}
 \begin{translit}{O. foobar XIV}{verso}{10-15}{LES 12,2-5}
 iw.i rx.kw mdw nTr
 \traduction
 I know the hieroglyphs
 \end{translit}

 \item the exemple environment. \sloppy
 a translit environment with the word ``Exemple'' in front of 
 it, and numbered. To write ``example'',  use
 \verb/\def\EXEMPLE{Example}/

\fussy
 these two environments write a line in ``\verb/*.dic/'' where \verb/*/
 is the name of your TeX file.

 ``translit'' writes:
\begin{verbatim}
 \Citation{name of the source}{references}{document's title}{
                        page number in document}
\end{verbatim}

 ``exemple'' writes:
\begin{verbatim}
 \Exemple{name of the source}{references}{document's title}
          {example number}{page number in document}
\end{verbatim}
(all on the same line)

\item the \verb/\traduction{}{}/ macro. It takes two arguments,
 the first being a transliteration, the second a translation.
 you have separates footnotes 
 and, by default, the text is given in two columns 
 which can spread over pages, which is useful to translate poetry.
 If you want the translation under  the transliteration, you
 can type 
         $$\hbox{\verb/\def\EcritTraduction{\EcritTraductionEnLigne}/}$$
 and 
         $$\hbox{\verb/\def\EcritTraduction{\EcritTraductionEnColonne}/}$$
 to switch back. In the column version, the second column is
 a verse-like environment
  
\end{enumerate}

\subsection{Varia}

  

\begin{description}
%\item[$<$ \rm and $>$] can be used to stress an error.  (in
%  \verb/\traduction{}{}/ only; prone to change)

                           
\item[\Backslash affligne] shows the line number above a vertical
  line: \affligne .

\item[\Backslash affpage] shows the page number in a cartouche:
  \affpage .

\item[\Backslash \tt | ] increases the line number and shows it.

\item[\Backslash \tt * ] increases the page number and shows it.  (the
  line number becomes 1)

\item[\Backslash numligne\{VALUE\}] gives a value to the line number
  and shows it

\item[\Backslash numpage\{VALUE\}] the same for page number.

\item[\Backslash dico\{Y\}\{translation\}\{comments\}] Can be used to make
  an index of terms. ex:
  $$\hbox{\verb| \dico{XAa}{to free}{transitive verb}|}$$  writes in the
  .dic file:
\begin{verbatim}
\DicoIndex {XAa}{ to free }{transitive verb }{P. Leyde I 350}
{verso,13}{name of the text}{2}
  
\end{verbatim}
that is, what you wrote, plus the references.  It is to be used
inside an environment. The first argument may contain a
\verb/hieroglyph/ environment.
\end{description}

\subsection{Grammatical signs and al.}

 The zero-subject (i.e. the empty set) is bound to \verb/\zero/

 There is a environment for typesetting grammatical rules:
 its name is gramrule.
 a word typed there appears in slanted font, and `\verb*/~X /' 
 writes X in transliteration (the space is mandatory).
 
 The possib environment allows to type different cases, with 
 an accolade in front of them. (\verb/\\/ to part the cases)
 
 The pile environment allows to write some text in a column.
 
 Example of use:
        
\begin{verbatim}
 \begin{gramrule} 
 ~ir + \begin{possib}
        infinitive\\
        prospective ~sDm.f \\
        ~mrr.f \\
        \end{possib} 
     + \pile{this is not\\to \\be taken seriously\\}
  \end{gramrule}
\end{verbatim}
 \begin{gramrule} 
 ~ir + \begin{possib}
        infinitive\\
        prospective ~sDm.f \\
        ~mrr.f \\
        \end{possib} 
     + \pile{this is not\\to \\be taken seriously\\}
  \end{gramrule}

\section{To Do}
\begin{itemize}
\item Capital letters for transliteration (currently, it is quite
  hard to type ``{\it R}\htimage{\eg a\/}''.
\item fully and strictly implement the ``manuel de codage''. (but
  waiting for a possible new version.)
\end{itemize}

\section{Recently Done}
\begin{itemize}
\item A right-to-left font, and a support for tex--xet.
\item Improved sesh --- I now use an hash table for the signs. The
  {\em manuel\/} is more or less supported, and should be so very quickly.
\item Hieroglyphic typesetting improved. The sizes are better choosed,
  the signs can be merged with text, and are now centered on the base
  line.
\item Can use NFSS2 now. (but still support the old system.)
\item Added hieroglyphs in \verb/\dico..../;
\item Added a possibility of reversed text in normal latex.
\item 2000AD. long time after. fixed rotation stuff.
\item cleaned up a new error for text in column. 
\item 2001. The monolith on the moon hasn't been found nor is HAL
  operational, yet there are two bugs less in Hiero\TeX{}: subgroup
  scaling is now correct, and a bug has been fixed in right-to-left
  text. Space (.) and quater-space (..) used to be swapped. New
  shading system. New signs can now be added without recompiling. The
  internal signs encoding was changed and simplified. To keep old dvi
  files usable, we decided to change the fonts names.
\item 2002 : a few days of work have : (a) fixed the shading system
  which works more or less perfectly in horizontal mode~; (b) fixed
  the ``philological signs'' system ([]...) so that it scales well
  when in subgroups (it was simply a matter of adding
  \texttt{\textbackslash scriptscriptstyle} in front of the \TeX{}
  commands).  Large shaded areas are now shaded group by group, which
  means the line-breaking algorithms work on them. Last but not least,
  postscript type 1 fonts have been made for the hieroglyphs, and the
  whole system works now with \texttt{pdflatex}.
\end{itemize}

\appendix{}
\section{Examples}

Here are a few examples, taken from various texts: Lef\`ebvre's {\it
  Grammaire de l'\'Egyptien Classique,\/} Gardiner's {\it Egyptian
  Grammar,\/} and Westcar.
\vspace{2em}

 \noindent{\bf \Large Le pseudoparticipe}

 \paragraph{} 
 Il exprime particuli\`erement l'{\bf \'etat} ou la {\bf condition} de qqn,
 ou de qq.ch. Deux cas sont \`a distinguer, selon que le substantif (ou
 pronom) auquel est appos\'e le pseudoparticipe est sujet ou objet du
 verbe de la phrase.

 \begin{enumerate}
 \item le pseudoparticipe peut qualifier un substantif (ou pronom) {\em
     sujet\/} de certains verbes, comme 
   \EnPetit{\begin{hieroglyph}{\leavevmode \Cadrat{\CadratLineI{\Aca GG/70/}\CadratLine{\Aca GD/52/}}\HinterSignsSpace
\Cadrat{\CadratLineI{\Aca GN/69/}\CadratLine{\Aca GN/36/}}}\end{hieroglyph}}
   \htimage{\eg wrS\/} `` passer tout le jour \`a '', 
   \EnPetit{\begin{hieroglyph}{\leavevmode \loneSign{\Aca GS/63/}\HinterSignsSpace
\Cadrat{\CadratLineI{\Aca GM/67/}\CadratLine{\Aca GD/52/}}\HinterSignsSpace
\loneSign{\Aca GA/89/}}\end{hieroglyph}}
 \htimage{\eg sDr\/} `` passer toute la nuit \`a '',  
   \EnPetit{\begin{hieroglyph}{\leavevmode \Cadrat{\CadratLineI{\Aca GO/32/}\CadratLine{\Aca GD/52/}}\HinterSignsSpace
\loneSign{\Aca GD/88/}}\end{hieroglyph}}
 \htimage{\eg pri\/} au sens de `` devenir '' ;
   \EnPetit{\begin{hieroglyph}{\leavevmode \Cadrat{\CadratLineI{\Aca GM/67/}\CadratLine{\Aca GD/52/}}}\end{hieroglyph}}
 \htimage{\eg Dr\/} `` finir '' (par faire telle ou telle chose).
   Ex.: 
   \EnPetit{\begin{hieroglyph}{\leavevmode \loneSign{\Aca GD/68/}\HwordSpace
\loneSign{\Aca GS/63/}\HinterSignsSpace
\Cadrat{\CadratLineI{\Aca GM/67/}\CadratLine{\Aca GD/52/}}\HinterSignsSpace
\Cadrat{\CadratLineI{\Aca GD/69/}\CadratLine{\Aca GA/89/}}\HwordSpace
\Cadrat{\CadratLineI{\Aca GO/65/}\CadratLine{\Aca GA/32/}}\HwordSpace
\loneSign{\Aca GV/59/}\HinterSignsSpace
\Cadrat{\CadratLineI{\Aca GN/60/}\CadratLine{\Aca GD/52/}}\HinterSignsSpace
\loneSign{\Aca GG/77/}\HinterSignsSpace
\Cadrat{\CadratLineI{\Aca GG/71/}\CadratLine{\Aca GD/52/}}\HwordSpace
\loneSign{\Aca GD/79/}\HinterSignsSpace
\loneSign{\Aca GW/52/}\HinterSignsSpace
\loneSign{\Aca GM/48/}\HinterSignsSpace
\loneSign{\Aca GN/54/}\HinterSignsSpace
\loneSign{\Aca GA/32/}}\end{hieroglyph}}
 \htimage{\eg n sDr s Hqrw r dmi.i\/} jamais un homme ne passa la nuit \`a
   avoir faim dans ma ville ({\it Menthuw.\/} 11). Litt. \`a l'\'etat
   d'avoir faim.

   S'il est sourd et que sa bouche ne puisse plus s'ouvrir, 
   \EnPetit{\begin{hieroglyph}{\leavevmode \Cadrat{\CadratLineI{\Aca GQ/34/}\CadratLine{\Aca GD/52/}}\HinterSignsSpace
\Cadrat{\CadratLineI{\Aca GD/52/}\CadratLine{\Aca GD/88/}}\HwordSpace
\Cadrat{\CadratLineI{\Aca GD/69/}\CadratLine{\Aca GX/32/\hfill\Aca GF/82/}}\HinterSignsSpace
\Cadrat{\CadratLineI{\Aca GZ/33/}\CadratLine{\Aca GI/41/}}\HwordSpace
\Cadrat{\CadratLineI{\Aca GV/61/}\CadratLine{\Aca GX/32/}}\HwordSpace
\loneSign{\Aca GM/53/}\HinterSignsSpace
\loneSign{\Aca GM/53/}\HinterSignsSpace
\Cadrat{\CadratLineI{\Aca GN/66/}\CadratLine{\Aca GN/66/}}\HinterSignsSpace
\loneSign{\Aca GA/38/}}\end{hieroglyph}}
   tous ses membres deviennent faibles ({\it Ebers\/} 99, 20--21).

   \EnPetit{\begin{hieroglyph}{\leavevmode \Cadrat{\CadratLineI{\Aca GM/67/}\CadratLine{\Aca GD/52/}}\HinterSignsSpace
\loneSign{\Aca GM/48/}\HinterSignsSpace
\Cadrat{\CadratLineI{\Aca GN/66/}\CadratLine{\Aca GI/41/}}\HwordSpace
\loneSign{\Aca GN/73/}\HinterSignsSpace
\loneSign{\Aca GS/63/}\HinterSignsSpace
\loneSign{\Aca GA/34/}}\end{hieroglyph}}
 \htimage{\eg Dr.in.f Hms(w)\/} \`a la fin il s'assit ({\it Leb.\/} 75).

   Comparer l'emploi, comme auxiliaires, devant \htimage{\eg sDm.n.f,\/} de \htimage{\eg
     pri, sDr\/} et \htimage{\eg Dr,} \S 331.

 \item Il peut qualifier un substantif [...]
 \end{enumerate}


 \noindent{\bf EXERCICE XXII}  

 \par ({\it a\/}) {\it Translate into English: \/}

 {\EnGros
 \begin{hieroglyph}{\leavevmode {\rm  (1) }\HwordSpace
\Cadrat{\CadratLineI{\Aca GD/79/}\CadratLine{\Aca GO/35/}}\HinterSignsSpace
\Cadrat{\CadratLineI{\Aca GN/66/}\CadratLine{\Aca GD/32/}}\HinterSignsSpace
\loneSign{\Aca GV/62/}\HinterSignsSpace
\loneSign{\Aca GG/77/}\HwordSpace
\loneSign{\Aca GD/52/}\HwordSpace
\loneSign{\Aca GF/51/}\HinterSignsSpace
\Cadrat{\CadratLineI{\Aca GO/82/}\CadratLine{\Aca GO/82/}}\HwordSpace
\Cadrat{\CadratLineI{\Aca GF/35/}\CadratLine{\Aca GD/69/}}\HwordSpace
\loneSign{\Aca GG/50/}\HwordSpace
\Cadrat{\CadratLineI{\Aca GO/80/}\CadratLine{\Aca GX/32/}}\HinterSignsSpace
\loneSign{\Aca GZ/32/}\HwordSpace
\loneSign{\Aca GF/51/}\HinterSignsSpace
\loneSign{\Aca GA/40/}\HinterSignsSpace
\loneSign{\Aca GX/32/}\HwordSpace
\loneSign{\Aca GG/50/}\HwordSpace
\loneSign{\Aca GM/48/}\HinterSignsSpace
\Cadrat{\CadratLineI{\Aca GQ/34/}\CadratLine{\Aca GX/32/}}\HwordSpace
\loneSign{\Aca GQ/32/}\HinterSignsSpace
\Cadrat{\CadratLineI{\Aca GX/32/}\CadratLine{\Aca GZ/33/}}\HinterSignsSpace
\loneSign{\Aca GO/80/}\HwordSpace
{\rm  (2) }\HwordSpace
\loneSign{\Aca GM/48/}\HinterSignsSpace
\loneSign{\Aca GG/77/}\HwordSpace
\loneSign{\Aca GG/62/}\HinterSignsSpace
\loneSign{\Aca GG/50/}\HinterSignsSpace
\Cadrat{\CadratLineI{\Aca GN/66/}\CadratLine{\Aca GI/41/}}\HwordSpace
\Cadrat{\CadratLineI{\Aca GN/73/}\CadratLine{\Aca GX/32/}}\HinterSignsSpace
\loneSign{\Aca GB/32/}\HwordSpace
\Cadrat{\CadratLineI{\Aca GX/32/}\CadratLine{\Aca GN/66/}}\HwordSpace
\Cadrat{\CadratLineI{\Aca GN/73/}\CadratLine{\Aca GO/65/}}\HinterSignsSpace
\loneSign{\Aca GA/34/}\HinterSignsSpace
\loneSign{\Aca GU/65/}\HinterSignsSpace
\loneSign{\Aca GM/48/}\HwordSpace
\Cadrat{\CadratLineI{\Aca GD/32/}\CadratLine{\Aca GZ/32/}}\HinterSignsSpace
\loneSign{\Aca GS/63/}\HwordSpace
\Cadrat{\CadratLineI{\Aca GD/33/}\CadratLine{\Aca GZ/32/}}\HwordSpace
\loneSign{\Aca GU/32/}\HinterSignsSpace
\negAROBspace\loneSign{\Aca GG/32/}\HinterSignsSpace
\loneSign{\Aca GQ/32/}\HinterSignsSpace
\loneSign{\Aca GX/32/}\HinterSignsSpace
\loneSign{\Aca GD/90/}\HinterSignsSpace
\loneSign{\Aca GS/63/}}\end{hieroglyph}
 }


 \noindent{\bf Un texte sym\'etrique}
%{\EgypS 1234567}
 \begin{center}

   \mbox{\begin{hieroglyph}{\leavevmode \boxrightleft{\loneSign{\Aca GS/68/}\HwordSpace
\loneSign{\Aca GX/40/}\HwordSpace
\cartouche{\loneSign{\Aca GL/32/}\HinterSignsSpace
\loneSign{\Aca GY/36/}\HinterSignsSpace
\loneSign{\Aca GN/36/}}%
\HwordSpace
\Cadrat{\CadratLineI{\Aca GL/33/}\CadratLine{\Aca GX/32/}}\HwordSpace
\Cadrat{\CadratLineI{\Aca GM/54/}\CadratLine{\Aca GX/32/}}\HwordSpace
\Cadrat{\CadratLineI{\Aca GX/32/}\CadratLine{\Aca GO/80/}}\HinterSignsSpace
\loneSign{\Aca GR/50/}\HwordSpace
\loneSign{\Aca GG/50/}\HwordSpace
\loneSign{\Aca GN/59/}\HwordSpace
\loneSign{\Aca GD/73/}\HwordSpace
\loneSign{\Aca GE/32/}\HwordSpace
\loneSign{\Aca GG/36/}\HwordSpace
\loneSign{\Aca GS/68/}}%
\HwordSpace
\loneSign{\Aca GG/36/}\HwordSpace
\loneSign{\Aca GE/32/}\HwordSpace
\loneSign{\Aca GD/73/}\HwordSpace
\loneSign{\Aca GN/59/}\HwordSpace
\loneSign{\Aca GG/50/}\HwordSpace
\loneSign{\Aca GR/50/}\HinterSignsSpace
\Cadrat{\CadratLineI{\Aca GX/32/}\CadratLine{\Aca GO/80/}}\HwordSpace
\Cadrat{\CadratLineI{\Aca GM/54/}\CadratLine{\Aca GX/32/}}\HwordSpace
\Cadrat{\CadratLineI{\Aca GL/33/}\CadratLine{\Aca GX/32/}}\HwordSpace
\cartouche{\loneSign{\Aca GN/36/}\HinterSignsSpace
\loneSign{\Aca GY/36/}\HinterSignsSpace
\loneSign{\Aca GL/32/}}%
\HwordSpace
\loneSign{\Aca GX/40/}\HwordSpace
\loneSign{\Aca GS/68/}}\end{hieroglyph}}
\end{center}

 \begin{hieroglyph}{\leavevmode \loneSign{\Aca GG/36/}\HwordSpace
\serekh{\loneSign{\Aca GE/32/}\HwordSpace
\loneSign{\Aca GD/73/}\HwordSpace
\loneSign{\Aca GN/59/}\HwordSpace
\loneSign{\Aca GG/50/}\HwordSpace
\LoneHorizontalLine{\loneSign{\Aca GR/50/}\HinterSignsSpace
\Cadrat{\CadratLineI{\Aca GX/32/}\CadratLine{\Aca GO/80/}}}}%
\HwordSpace
\loneSign{\Aca GG/49/}\HwordSpace
\loneSign{\Aca GV/60/}\HwordSpace
\LoneHorizontalLine{\loneSign{\Aca GM/54/}\HinterSignsSpace
\loneSign{\Aca GX/32/}}\HinterSignsSpace
\loneSign{\Aca GM/48/}\HinterSignsSpace
\loneSign{\Aca GM/48/}\HwordSpace
\LoneHorizontalLine{\Cadrat{\CadratLineI{\Aca GN/36/}\CadratLine{\Aca GZ/32/}}\HinterSignsSpace
\loneSign{\Aca GW/52/}}\HwordSpace
\loneSign{\Aca GG/50/}\HwordSpace
\Cadrat{\CadratLineI{\Aca GQ/34/\hfill\Aca GX/32/}\CadratLine{\Aca GN/32/}}\HwordSpace
\LoneHorizontalLine{\loneSign{\Aca GS/77/}\HinterSignsSpace
\loneSign{\Aca GG/41/}\HinterSignsSpace
\Cadrat{\CadratLineI{\Aca GF/40/}\CadratLine{\Aca GF/40/}}}\HinterSignsSpace
\LoneHorizontalLine{\Cadrat{\CadratLineI{\Aca GD/78/}\CadratLine{\Aca GN/59/}}\HinterSignsSpace
\loneSign{\Aca GZ/34/}}\HwordSpace
\LoneHorizontalLine{\Cadrat{\CadratLineI{\Aca GM/54/}\CadratLine{\Aca GX/32/}}\HinterSignsSpace
\Cadrat{\CadratLineI{\Aca GL/33/}\CadratLine{\Aca GX/32/}}}\HinterSignsSpace
\cartouche{\loneSign{\Aca GN/36/}\HinterSignsSpace
\loneSign{\Aca GY/36/}\HinterSignsSpace
\loneSign{\Aca GL/32/}}%
\HwordSpace
\LoneHorizontalLine{\loneSign{\Aca GG/73/}\HinterSignsSpace
\negAROBspace\negAROBspace\negAROBspace\Cadrat{\CadratLineI{\Aca GN/36/}\CadratLine{\HquarterSpace }}}\HinterSignsSpace
\cartouche{\loneSign{\Aca GG/59/}\HinterSignsSpace
\LoneHorizontalLine{\loneSign{\Aca GF/62/}\HinterSignsSpace
\loneSign{\Aca GF/66/}}\HinterSignsSpace
\loneSign{\Aca GL/32/}}%
\HwordSpace
\loneSign{\Aca GO/41/}\HwordSpace
\Cadrat{\CadratLineI{\Aca GV/61/}\CadratLine{\Aca GX/32/}}\HwordSpace
\Cadrat{\CadratLineI{\Aca GAa/43/}\CadratLine{\Aca GI/41/}}\HwordSpace
\LoneHorizontalLine{\Cadrat{\CadratLineI{\Aca GD/60/}\CadratLine{\Aca GX/32/}}\HinterSignsSpace
\Cadrat{\CadratLineI{\Aca GN/64/}\CadratLine{\Aca GN/64/}\CadratLine{\Aca GN/64/}}}\HinterSignsSpace
\LoneHorizontalLine{\loneSign{\Aca GU/37/}\HinterSignsSpace
\loneSign{\Aca GM/48/}\HinterSignsSpace
\loneSign{\Aca GM/48/}}}\end{hieroglyph}

 \begin{center}
   \EnColonne[1.2\Htm]{
     \begin{hieroglyph}{\leavevmode \loneSign{\Aca GG/36/}\HwordSpace
\serekh{\loneSign{\Aca GE/32/}\HwordSpace
\loneSign{\Aca GD/73/}\HwordSpace
\loneSign{\Aca GN/59/}\HwordSpace
\loneSign{\Aca GG/50/}\HwordSpace
\LoneHorizontalLine{\loneSign{\Aca GR/50/}\HinterSignsSpace
\Cadrat{\CadratLineI{{\Hsmaller\Aca GX/32/}}\CadratLine{{\Hsmaller\Aca GO/80/}}}}}%
}\end{hieroglyph}\HendOfLine
\begin{hieroglyph}{\leavevmode
\loneSign{\Aca GG/49/}\HwordSpace
\loneSign{\Aca GV/60/}\HwordSpace
\LoneHorizontalLine{\loneSign{\Aca GM/54/}\HinterSignsSpace
\negAROBspace\loneSign{{\Hsmaller\Aca GX/32/}}\HinterSignsSpace
\negAROBspace\loneSign{\Aca GM/48/}\HinterSignsSpace
\loneSign{\Aca GM/48/}}\HwordSpace
\LoneHorizontalLine{\Cadrat{\CadratLineI{\Aca GN/36/}\CadratLine{\Aca GZ/32/}}\HinterSignsSpace
\loneSign{\Aca GW/52/}}\HwordSpace
\loneSign{\Aca GG/50/}\HwordSpace
\Cadrat{\CadratLineI{\Aca GQ/34/\hfill\Aca GX/32/}\CadratLine{\Aca GN/32/}}}\end{hieroglyph}\HendOfLine
\begin{hieroglyph}{\leavevmode
\LoneHorizontalLine{\loneSign{\Aca GS/77/}\HinterSignsSpace
\loneSign{\Aca GG/41/}}\HwordSpace
\LoneHorizontalLine{\loneSign{\Aca GF/40/}\HinterSignsSpace
\loneSign{\Aca GF/40/}}\HwordSpace
\loneSign{\Aca GD/78/}\HinterSignsSpace
\loneSign{\Aca GN/59/}\HinterSignsSpace
\loneSign{\Aca GZ/33/}}\end{hieroglyph}\HendOfLine
\begin{hieroglyph}{\leavevmode
\LoneHorizontalLine{\Cadrat{\CadratLineI{\Aca GM/54/}\CadratLine{\Aca GX/32/}}\HinterSignsSpace
\Cadrat{\CadratLineI{\Aca GL/33/}\CadratLine{\Aca GX/32/}}}\HinterSignsSpace
\cartouche{\loneSign{\Aca GN/36/}\HinterSignsSpace
\loneSign{\Aca GY/36/}\HinterSignsSpace
\loneSign{\Aca GL/32/}}%
}\end{hieroglyph}\HendOfLine
\begin{hieroglyph}{\leavevmode
\LoneHorizontalLine{\loneSign{\Aca GG/73/}\HinterSignsSpace
\negAROBspace\negAROBspace\negAROBspace\Cadrat{\CadratLineI{\Aca GN/36/}\CadratLine{\HquarterSpace }}}\HinterSignsSpace
\cartouche{\loneSign{\Aca GG/59/}\HinterSignsSpace
\LoneHorizontalLine{\loneSign{\Aca GF/62/}\HinterSignsSpace
\loneSign{\Aca GF/66/}}\HinterSignsSpace
\loneSign{\Aca GL/32/}}%
}\end{hieroglyph}\HendOfLine
\begin{hieroglyph}{\leavevmode
\loneSign{\Aca GO/41/}\HwordSpace
\loneSign{\Aca GV/61/}\HinterSignsSpace
\loneSign{\Aca GX/32/}\HwordSpace
\loneSign{\Aca GAa/43/}\HinterSignsSpace
\loneSign{\Aca GI/41/}\HwordSpace
\LoneHorizontalLine{\Cadrat{\CadratLineI{\Aca GD/60/}\CadratLine{\Aca GX/32/}}\HinterSignsSpace
\Cadrat{\CadratLineI{\Aca GN/64/}\CadratLine{\Aca GN/64/}\CadratLine{\Aca GN/64/}}}\HinterSignsSpace
\LoneHorizontalLine{\loneSign{\Aca GU/37/}\HinterSignsSpace
\loneSign{\Aca GM/48/}\HinterSignsSpace
\loneSign{\Aca GM/48/}}}\end{hieroglyph}}
 \end{center}




 \noindent{\bf Un extrait de Westcar} 

 {\EnGros\SurLigne
   \begin{hieroglyph}{\leavevmode \enrouge{\loneSign{\Aca GP/38/}\HinterSignsSpace
\Cadrat{\CadratLineI{\Aca GD/69/}\CadratLine{\Aca GD/88/}}\HwordSpace
\loneSign{\Aca GQ/34/}\HinterSignsSpace
\loneSign{\Aca GZ/40/}\HwordSpace
\Cadrat{\CadratLineI{\Aca GD/35/}\CadratLine{\Aca GN/66/}}\HwordSpace
\loneSign{\Aca GM/54/}\HinterSignsSpace
\Cadrat{\CadratLineI{\Aca GX/32/}\CadratLine{\Aca GN/66/}}\HwordSpace
\loneSign{\Aca GG/73/}\HwordSpace
\loneSign{\Aca GG/36/}\HinterSignsSpace
\Cadrat{\CadratLineI{\Aca GD/70/}\CadratLine{\Aca GD/70/}}\HinterSignsSpace
\loneSign{\Aca GI/41/}\HinterSignsSpace
\loneSign{\Aca GA/85/}\HwordSpace
\loneSign{\Aca GD/52/}\HwordSpace
\loneSign{\Aca GS/78/}\HinterSignsSpace
\Cadrat{\CadratLineI{\Aca GD/79/}\CadratLine{\Aca GX/32/}}\HinterSignsSpace
\loneSign{\Aca GA/33/}}%
\HinterSignsSpace
\enrouge{\loneSign{\ligAROBDd }\HinterSignsSpace
\loneSign{\Aca GI/41/}}%
\HinterSignsSpace
\hachure{\HfullSpace \HinterSignsSpace
\loneSign{\Aca GA/33/}\HinterSignsSpace
\loneSign{\Aca GN/66/}\HinterSignsSpace
\Cadrat{\CadratLineI{\Aca GO/65/}\CadratLine{\Aca GQ/34/\hfill\HquarterSpace }}\HinterSignsSpace
\HfullSpace \HinterSignsSpace
\HfullSpace }%
\HinterSignsSpace
\HquarterSpace \HinterSignsSpace
\hachure{\HfullSpace \HinterSignsSpace
\HfullSpace }%
\HinterSignsSpace
\loneSign{\Aca GG/50/}\HwordSpace
\Cadrat{\CadratLineI{\Aca GD/52/}\CadratLine{\Aca GAa/32/}}\HinterSignsSpace
\Cadrat{\CadratLineI{\Aca GX/32/}\CadratLine{\Aca GY/32/}}\HwordSpace
\loneSign{\Aca GN/66/}\HwordSpace
\Cadrat{\CadratLineI{\Aca GN/66/}\CadratLine{\Aca GX/32/}}\HinterSignsSpace
\loneSign{\Aca GG/35/}\HinterSignsSpace
\loneSign{\Aca GZ/35/}\HwordSpace
\Cadrat{\CadratLineI{\Aca GZ/44/}\CadratLine{\Aca GD/88/}}\HwordSpace
\loneSign{\Aca GD/68/}\HwordSpace
\Cadrat{\CadratLineI{\Aca GD/52/}\CadratLine{\Aca GAa/32/}}\HinterSignsSpace
\Cadrat{\CadratLineI{\Aca GY/32/}\CadratLine{\Aca GN/66/}}\HinterSignsSpace
\loneSign{\Aca GX/32/}\HinterSignsSpace
\loneSign{\Aca GZ/40/}\HwordSpace
\Cadrat{\CadratLineI{\Aca GU/32/}\CadratLine{\Aca GAa/41/}}\HinterSignsSpace
\Cadrat{\CadratLineI{\Aca GD/69/}\CadratLine{\Aca GX/32/\hfill\Aca GZ/38/}}\HinterSignsSpace
\Cadrat{\CadratLineI{\Aca GY/32/}\CadratLine{\Aca GZ/33/}}\HwordSpace
\loneSign{\Aca GD/52/}\HwordSpace
\Cadrat{\CadratLineI{\Aca GW/44/}\CadratLine{\Aca GD/52/}}\HinterSignsSpace
\loneSign{\Aca GU/49/}\HwordSpace
\loneSign{\Aca GM/48/}\HinterSignsSpace
\loneSign{\Aca GZ/40/}\HwordSpace
\Cadrat{\CadratLineI{\Aca GE/66/}\CadratLine{\Aca GN/66/}}\HwordSpace
\Cadrat{\CadratLineI{\Aca GAa/32/}\CadratLine{\Aca GD/52/}}\HwordSpace
\loneSign{\Aca GU/70/}\HinterSignsSpace
\loneSign{\Aca GZ/32/}\HinterSignsSpace
\loneSign{\Aca GG/38/}\HinterSignsSpace
\loneSign{\Aca GV/62/}\HwordSpace
\loneSign{\Aca GG/50/}\HwordSpace
\loneSign{\Aca GO/35/}\HinterSignsSpace
\loneSign{\Aca GG/32/}\HinterSignsSpace
\loneSign{\Aca GZ/40/}\HinterSignsSpace
\Cadrat{\CadratLineI{\Aca GY/32/}\CadratLine{\Aca GZ/33/}}\HinterSignsSpace
\loneSign{\Aca GV/63/}\HwordSpace
\Cadrat{\CadratLineI{\ligAROBD}\CadratLine{\Aca GO/65/}}\HinterSignsSpace
\loneSign{\Aca GV/63/}\HwordSpace
\loneSign{\ligAROBDd }\HinterSignsSpace
\loneSign{\Aca GM/48/}\HinterSignsSpace
\loneSign{\Aca GN/66/}\HwordSpace
\loneSign{\Aca GU/70/}\HinterSignsSpace
\loneSign{\Aca GZ/32/}\HinterSignsSpace
\loneSign{\Aca GG/38/}\HinterSignsSpace
\loneSign{\Aca GI/41/}\HwordSpace
\loneSign{\Aca GM/48/}\HinterSignsSpace
\Cadrat{\CadratLineI{\Aca GV/37/}\CadratLine{\Aca GO/65/}}\HinterSignsSpace
\loneSign{\Aca GX/32/}\HinterSignsSpace
\loneSign{\Aca GA/33/}\HwordSpace
\loneSign{\Aca GQ/34/}\HinterSignsSpace
\loneSign{\Aca GZ/40/}\HwordSpace
\loneSign{\Aca GG/36/}\HinterSignsSpace
\Cadrat{\CadratLineI{\Aca GD/70/}\CadratLine{\Aca GD/70/}}\HinterSignsSpace
\loneSign{\Aca GI/41/}\HinterSignsSpace
\loneSign{\Aca GA/85/}\HwordSpace
\loneSign{\Aca GG/73/}\HinterSignsSpace
\loneSign{\Aca GG/38/}\HinterSignsSpace
\loneSign{\Aca GA/32/}\HwordSpace
\loneSign{\ligAROBDd }\HinterSignsSpace
\loneSign{\Aca GM/48/}\HinterSignsSpace
\loneSign{\Aca GN/66/}\HwordSpace
\loneSign{\Aca GM/54/}\HinterSignsSpace
\Cadrat{\CadratLineI{\Aca GX/32/}\CadratLine{\Aca GN/66/}}\HwordSpace
\loneSign{\Aca GG/73/}\HwordSpace
\loneSign{\Aca GG/36/}\HinterSignsSpace
\Cadrat{\CadratLineI{\Aca GD/70/}\CadratLine{\Aca GD/70/}}\HinterSignsSpace
\loneSign{\Aca GI/41/}\HinterSignsSpace
\loneSign{\Aca GA/85/}\HwordSpace
\loneSign{\Aca GM/48/}\HinterSignsSpace
\loneSign{\Aca GZ/40/}\HwordSpace
\Cadrat{\CadratLineI{\Aca GE/66/}\CadratLine{\Aca GN/66/}}\HwordSpace
\Cadrat{\CadratLineI{\Aca GN/66/}\CadratLine{\ligAROBD}}\HinterSignsSpace
\loneSign{\Aca GS/63/}\HinterSignsSpace
\Cadrat{\CadratLineI{\Aca GG/71/}\CadratLine{\Aca GA/32/}}\HwordSpace
\loneSign{\Aca GR/42/}\HinterSignsSpace
\loneSign{\Aca GR/42/}\HinterSignsSpace
\loneSign{\Aca GM/48/}\HinterSignsSpace
\Cadrat{\CadratLineI{\Aca GY/32/}\CadratLine{\Aca GA/32/}}\HwordSpace
\Cadrat{\CadratLineI{\Aca GD/52/}\CadratLine{\Aca GN/66/}}\HinterSignsSpace
\loneSign{\Aca GA/33/}\HinterSignsSpace
\loneSign{\Aca GI/41/}\HwordSpace
\Cadrat{\CadratLineI{\Aca GN/73/}\CadratLine{\Aca GO/65/}}\HinterSignsSpace
\loneSign{\Aca GA/38/}\HinterSignsSpace
\loneSign{\Aca GI/41/}\HwordSpace
\loneSign{\Aca GG/50/}\HwordSpace
\loneSign{\Aca GR/42/}\HinterSignsSpace
\loneSign{\Aca GR/42/}\HinterSignsSpace
\loneSign{\Aca GY/32/}\HwordSpace
\cartouche{\loneSign{\Aca GS/63/}\HinterSignsSpace
\loneSign{\Aca GF/66/}\HinterSignsSpace
\Cadrat{\CadratLineI{\Aca GI/41/}\CadratLine{\Aca GD/52/}}\HinterSignsSpace
\loneSign{\Aca GZ/40/}}%
\HwordSpace
\Cadrat{\CadratLineI{\Aca GU/32/}\CadratLine{\Aca GAa/41/}}\HwordSpace
\loneSign{\Aca GP/40/}\HinterSignsSpace
\loneSign{\Aca GZ/40/}\HinterSignsSpace
\loneSign{\Aca GA/33/}\HwordSpace
\loneSign{\Aca GM/48/}\HinterSignsSpace
\Cadrat{\CadratLineI{\Aca GZ/40/}\CadratLine{\Aca GI/41/}}\HwordSpace
\loneSign{\Aca GG/50/}\HwordSpace
\Cadrat{\CadratLineI{\Aca GN/66/}\CadratLine{\ligAROBD}}\HinterSignsSpace
\loneSign{\Aca GS/63/}\HinterSignsSpace
\Cadrat{\CadratLineI{\Aca GG/71/}\CadratLine{\Aca GA/32/}}\HwordSpace
\loneSign{\Aca GN/66/}\HwordSpace
\loneSign{\Aca GM/35/}\HinterSignsSpace
\Cadrat{\CadratLineI{\Aca GX/32/}\CadratLine{\Aca GV/32/}}\HinterSignsSpace
\loneSign{\Aca GV/51/}\HwordSpace
\loneSign{\Aca GM/48/}\HinterSignsSpace
\Cadrat{\CadratLineI{\Aca GZ/40/}\CadratLine{\Aca GI/41/}}\HwordSpace
\loneSign{\Aca GD/33/}\HinterSignsSpace
\loneSign{\Aca GZ/32/}\HwordSpace
\loneSign{\Aca GM/73/}\HinterSignsSpace
\loneSign{\Aca GG/50/}\HinterSignsSpace
\loneSign{\Aca GA/33/}\HwordSpace
\Cadrat{\CadratLineI{\Aca GX/32/}\CadratLine{\Aca GX/34/}}\HinterSignsSpace
\Cadrat{\CadratLineI{\Aca GN/49/}\CadratLine{\Aca GZ/33/}}\HwordSpace
\Cadrat{\CadratLineI{\Aca GV/32/\hfill\Aca GV/32/\hfill\Aca GV/32/}\CadratLine{\hierCC}}\HwordSpace
\loneSign{\Aca GD/52/}\HinterSignsSpace
\Cadrat{\CadratLineI{\Aca GY/36/}\CadratLine{\Aca GN/66/}}\HinterSignsSpace
\Cadrat{\CadratLineI{\Aca GD/74/}\CadratLine{\Aca GF/82/}}\HwordSpace
\loneSign{\Aca GN/66/}\HwordSpace
\loneSign{\Aca GE/32/}\HinterSignsSpace
\loneSign{\Aca GZ/32/}\HwordSpace
\loneSign{\Aca GG/50/}\HwordSpace
\loneSign{\Aca GM/48/}\HinterSignsSpace
\Cadrat{\CadratLineI{\Aca GZ/40/}\CadratLine{\Aca GI/41/}}\HinterSignsSpace
\Cadrat{\CadratLineI{\Aca GF/82/}\CadratLine{\Aca GZ/33/}}\HwordSpace
\loneSign{\Aca GV/59/}\HinterSignsSpace
\Cadrat{\CadratLineI{\Aca GN/66/}\CadratLine{\Aca GD/69/}}\HwordSpace
\loneSign{\Aca GS/63/}\HinterSignsSpace
\Cadrat{\CadratLineI{\Aca GG/70/}\CadratLine{\Aca GD/52/}}\HinterSignsSpace
\loneSign{\Aca GM/48/}\HinterSignsSpace
\loneSign{\Aca GN/67/}\HinterSignsSpace
\loneSign{\Aca GA/33/}\HwordSpace
\loneSign{\Aca GV/59/}\HinterSignsSpace
\Cadrat{\CadratLineI{\Aca GN/60/}\CadratLine{\Aca GX/32/}}\HinterSignsSpace
\Cadrat{\CadratLineI{\Aca GW/55/}\CadratLine{\Aca GZ/33/}}\HwordSpace
\Cadrat{\CadratLineI{\Aca GD/79/}\CadratLine{\Aca GO/65/}}\HinterSignsSpace
\loneSign{\Aca GW/55/}\HwordSpace
\loneSign{\Aca GV/32/}\HwordSpace
\loneSign{\Aca GD/52/}\HwordSpace
\Cadrat{\CadratLineI{\Aca GY/36/}\CadratLine{\Aca GN/66/}}\HinterSignsSpace
\loneSign{\Aca GY/32/}\HwordSpace
\loneSign{\Aca GG/50/}\HwordSpace
\loneSign{\Aca GO/35/}\HinterSignsSpace
\loneSign{\Aca GD/52/}\HinterSignsSpace
\loneSign{\Aca GZ/40/}\HinterSignsSpace
\loneSign{\Aca GN/36/}\HinterSignsSpace
\loneSign{\Aca GZ/32/}\HwordSpace
\Cadrat{\CadratLineI{\Aca GQ/34/}\CadratLine{\Aca GN/66/}}\HwordSpace
\loneSign{\Aca GM/48/}\HinterSignsSpace
\Cadrat{\CadratLineI{\Aca GZ/40/}\CadratLine{\Aca GI/41/}}\HwordSpace
\Cadrat{\CadratLineI{\Aca GD/52/}\CadratLine{\Aca GAa/32/}}\HinterSignsSpace
\loneSign{\Aca GY/32/}\HwordSpace
\loneSign{\Aca GS/57/}\HinterSignsSpace
\Cadrat{\CadratLineI{\Aca GV/32/}\CadratLine{\Aca GD/73/}}\HwordSpace
\loneSign{\Aca GD/32/}\HinterSignsSpace
\loneSign{\Aca GZ/32/}\HwordSpace
\loneSign{\Aca GV/59/}\HinterSignsSpace
\loneSign{\Aca GS/63/}\HinterSignsSpace
\loneSign{\Aca GN/60/}\HinterSignsSpace
\Cadrat{\CadratLineI{\Aca GZ/44/}\CadratLine{\Aca GD/73/}}\HwordSpace
\loneSign{\Aca GM/48/}\HinterSignsSpace
\Cadrat{\CadratLineI{\Aca GZ/40/}\CadratLine{\Aca GI/41/}}\HwordSpace
\Cadrat{\CadratLineI{\Aca GD/52/}\CadratLine{\Aca GAa/32/}}\HinterSignsSpace
\loneSign{\Aca GY/32/}\HwordSpace
\Cadrat{\CadratLineI{\Aca GX/32/}\CadratLine{\Aca GN/66/}}\HinterSignsSpace
\loneSign{\Aca GW/57/}\HinterSignsSpace
\loneSign{\Aca GZ/40/}\HinterSignsSpace
\loneSign{\Aca GT/49/}\HinterSignsSpace
\loneSign{\Aca GG/75/}\HinterSignsSpace
\Cadrat{\CadratLineI{\Aca GY/32/}\CadratLine{\Aca GZ/33/}}\HwordSpace
\Cadrat{\CadratLineI{\Aca GO/76/}\CadratLine{\Aca GX/32/\hfill\Aca GO/32/}}\HinterSignsSpace
\loneSign{\Aca GZ/35/}\HwordSpace
\Cadrat{\CadratLineI{\Aca GN/66/}\CadratLine{\Aca GX/32/}}\HwordSpace
\Cadrat{\CadratLineI{\Aca GE/66/}\CadratLine{\Aca GN/66/}}\HinterSignsSpace
\Cadrat{\CadratLineI{\Aca GX/32/}\CadratLine{\Aca GO/32/}}\HwordSpace
\Cadrat{\CadratLineI{\Aca GN/66/}\CadratLine{\Aca GX/32/}}\HwordSpace
\Cadrat{\CadratLineI{\Aca GG/59/}\CadratLine{\Aca GX/32/\hfill\Aca GZ/37/}}}\end{hieroglyph}
 }

Wadi Hammamat text M191, sent to me by J. KRAUS:
\begin{hieroglyph}{\leavevmode \traittexte{1}\HinterSignsSpace
\Cadrat{\CadratLineI{\Aca GM/54/}\CadratLine{\Aca GX/32/}}\HinterSignsSpace
\Cadrat{\CadratLineI{\Aca GL/33/}\CadratLine{\Aca GX/32/}}\HwordSpace
\cartouche{\loneSign{\Aca GN/36/}\HinterSignsSpace
\loneSign{\Aca GV/61/}\HinterSignsSpace
\Cadrat{\CadratLineI{\Aca GN/48/}\CadratLine{\Aca GN/48/}}}%
\HwordSpace
\loneSign{\Aca GS/68/}\HwordSpace
\Cadrat{\CadratLineI{\ligAROBDt}\CadratLine{\Aca GN/48/}}\HwordSpace
\loneSign{\Aca GF/62/}\HwordSpace
\loneSign{\Aca GN/66/}\HwordSpace
\loneSign{\Aca GM/54/}\HinterSignsSpace
\loneSign{\Aca GX/32/}\HinterSignsSpace
\loneSign{\Aca GG/47/}\HwordSpace
\loneSign{\Aca GM/48/}\HinterSignsSpace
\loneSign{\Aca GD/70/}\HinterSignsSpace
\loneSign{\Aca GG/50/}\HinterSignsSpace
\loneSign{\Aca GA/84/}\HwordSpace
\Cadrat{\CadratLineI{\Aca GN/42/}\CadratLine{\Aca GZ/32/\hfill\Aca GZ/32/}}\HinterSignsSpace
\Cadrat{\CadratLineI{\Aca GM/39/}\CadratLine{\Aca GAa/32/\hfill\Aca GX/32/}}\HinterSignsSpace
\loneSign{\Aca GN/36/}\HinterSignsSpace
\Cadrat{\CadratLineI{\Aca GV/51/\hfill\Aca GV/51/}\CadratLine{\Aca GZ/33/}}\HwordSpace
\loneSign{\Aca GG/77/}\HinterSignsSpace
\Cadrat{\CadratLineI{\Aca GD/79/}\CadratLine{\Aca GX/32/}\CadratLine{\Aca GD/69/}}\HwordSpace
\loneSign{\Aca GG/50/}\HwordSpace
\Cadrat{\CadratLineI{\Aca GD/60/}\CadratLine{\Aca GZ/32/}}\HinterSignsSpace
\Cadrat{\CadratLineI{\Aca GX/32/}\CadratLine{\Aca GZ/33/}}\HwordSpace
\traittexte{2}\HinterSignsSpace
\loneSign{\Aca GG/50/}\HwordSpace
\Cadrat{\CadratLineI{\Aca GN/57/}\CadratLine{\Aca GZ/32/}}\HwordSpace
\Cadrat{\CadratLineI{\Aca GQ/34/}\CadratLine{\Aca GN/66/}}\HwordSpace
\loneSign{\Aca GG/50/}\HwordSpace
\Cadrat{\CadratLineI{\Aca GN/69/}\CadratLine{\Aca GZ/32/}}\HwordSpace
\loneSign{\Aca GV/35/}\HinterSignsSpace
\loneSign{\Aca GG/32/}\HinterSignsSpace
\loneSign{\Aca GV/59/}\HinterSignsSpace
\loneSign{\Aca GV/60/}\HwordSpace
\loneSign{\Aca GV/61/}\HwordSpace
\loneSign{\Aca GS/68/}\HinterSignsSpace
\Cadrat{\CadratLineI{\Aca GN/66/}\CadratLine{\Aca GAa/32/}}\HwordSpace
\loneSign{\Aca GF/56/}\HinterSignsSpace
\loneSign{\Aca GG/50/}\HwordSpace
\loneSign{\Aca GD/92/}\HinterSignsSpace
\loneSign{\Aca GM/48/}\HinterSignsSpace
\loneSign{\Aca GG/32/}\HinterSignsSpace
\Cadrat{\CadratLineI{\Aca GX/32/}\CadratLine{\Aca GF/49/}}\HwordSpace
\Cadrat{\CadratLineI{\Aca GD/35/}\CadratLine{\Aca GX/32/}}\HwordSpace
\loneSign{\Aca GV/59/}\HinterSignsSpace
\loneSign{\Aca GG/77/}\HwordSpace
\Cadrat{\CadratLineI{\Aca GU/33/}\CadratLine{\Aca GD/35/}}\HinterSignsSpace
\loneSign{\Aca GG/32/}\HinterSignsSpace
\loneSign{\Aca GG/32/}\HinterSignsSpace
\Cadrat{\CadratLineI{\Aca GL/32/}\CadratLine{\Aca GD/52/}}\HinterSignsSpace
\loneSign{\Aca GG/77/}\HinterSignsSpace
\loneSign{\Aca GZ/34/}\HinterSignsSpace
\Cadrat{\CadratLineI{\Aca GW/57/}\CadratLine{\Aca GZ/32/}}\HinterSignsSpace
\loneSign{\Aca GR/39/}\HinterSignsSpace
\Cadrat{\CadratLineI{\Aca GQ/34/}\CadratLine{\Aca GN/66/}}\HinterSignsSpace
\traittexte{3}\HinterSignsSpace
\Cadrat{\CadratLineI{\Aca GD/70/}\CadratLine{\Aca GX/32/}}\HinterSignsSpace
\loneSign{\Aca GG/64/}\HinterSignsSpace
\Cadrat{\CadratLineI{\Aca GI/41/}\CadratLine{\Aca GN/66/}}\HinterSignsSpace
\loneSign{\Aca GG/56/}\HinterSignsSpace
\Cadrat{\CadratLineI{\Aca GX/32/}\CadratLine{\Aca GZ/33/}}\HinterSignsSpace
\Cadrat{\CadratLineI{\Aca GD/35/}\CadratLine{\Aca GX/32/}}\HinterSignsSpace
\Cadrat{\CadratLineI{\Aca GN/56/}\CadratLine{\Aca GX/32/\hfill\Aca GZ/32/}}\HinterSignsSpace
\loneSign{\Aca GG/50/}\HinterSignsSpace
\Cadrat{\CadratLineI{\Aca GU/52/}\CadratLine{\Aca GW/57/}}\HinterSignsSpace
\loneSign{\Aca GM/48/}\HinterSignsSpace
\loneSign{\Aca GM/48/}\HinterSignsSpace
\loneSign{\Aca GN/68/}\HwordSpace
\loneSign{\Aca GD/92/}\HinterSignsSpace
\loneSign{\Aca GS/63/}\HinterSignsSpace
\Cadrat{\CadratLineI{\Aca GX/32/}\CadratLine{\Aca GD/88/}}\HinterSignsSpace
\loneSign{\Aca GN/67/}\HinterSignsSpace
\Cadrat{\CadratLineI{\Aca GD/33/}\CadratLine{\Aca GZ/32/}}\HinterSignsSpace
\Cadrat{\CadratLineI{\Aca GN/66/}\CadratLine{\Aca GV/59/\hfill\Aca GG/32/}}\HinterSignsSpace
\Cadrat{\CadratLineI{\Aca GAa/33/}\CadratLine{\Aca GN/66/}}\HinterSignsSpace
\Cadrat{\CadratLineI{\Aca GN/69/}\CadratLine{\Aca GZ/32/}}\HinterSignsSpace
\loneSign{\Aca GG/62/}\HinterSignsSpace
\loneSign{\Aca GX/32/}\HinterSignsSpace
\Cadrat{\CadratLineI{\Aca GW/40/\hfill\Aca GX/32/}\CadratLine{\Aca GV/61/}}\HinterSignsSpace
\loneSign{\Aca GG/50/}\HinterSignsSpace
\Cadrat{\CadratLineI{\Aca GD/33/}\CadratLine{\Aca GF/65/}}\HinterSignsSpace
\Cadrat{\CadratLineI{\Aca GK/32/}\CadratLine{\Aca GN/66/}}\HinterSignsSpace
\Cadrat{\CadratLineI{\Aca GX/32/}\CadratLine{\Aca GN/56/}}\HinterSignsSpace
\traittexte{4}\HinterSignsSpace
\Cadrat{\CadratLineI{\Aca GV/53/}\CadratLine{\Aca GZ/32/}\CadratLine{\Aca GD/69/}}\HinterSignsSpace
\loneSign{\Aca GV/51/}\HinterSignsSpace
\loneSign{\Aca GD/52/}\HinterSignsSpace
\Cadrat{\CadratLineI{\Aca GV/53/}\CadratLine{\Aca GZ/32/}\CadratLine{\Aca GD/69/}}\HinterSignsSpace
\loneSign{\Aca GV/51/}\HinterSignsSpace
\Cadrat{\CadratLineI{\Aca GD/33/\hfill\Aca GZ/32/}\CadratLine{\Aca GD/52/}}\HinterSignsSpace
\loneSign{\Aca GS/63/}\HwordSpace
\loneSign{\Aca GV/61/}\HinterSignsSpace
\Cadrat{\CadratLineI{\Aca GV/53/}\CadratLine{\Aca GX/32/}\CadratLine{\Aca GY/33/}}\HinterSignsSpace
\loneSign{\Aca GG/50/}\HinterSignsSpace
\loneSign{\Aca GN/67/}\HinterSignsSpace
\loneSign{\Aca GD/52/}\HinterSignsSpace
\Cadrat{\CadratLineI{\Aca GN/66/}\CadratLine{\Aca GQ/34/}\CadratLine{\Aca GD/52/}}\HinterSignsSpace
\Cadrat{\CadratLineI{\Aca GX/32/\hfill\Aca GN/54/}\CadratLine{\Aca GZ/33/}}\HinterSignsSpace
\loneSign{\Aca GS/63/}\HinterSignsSpace
\loneSign{\Aca GS/63/}\HinterSignsSpace
\loneSign{\Aca GD/94/}\HinterSignsSpace
\loneSign{\Aca GX/32/}\HinterSignsSpace
\loneSign{\Aca GS/63/}\HinterSignsSpace
\Cadrat{\CadratLineI{\Aca GX/32/}\CadratLine{\Aca GG/70/}\CadratLine{\Aca GD/52/}}\HinterSignsSpace
\loneSign{\Aca GU/65/}\HinterSignsSpace
\loneSign{\Aca GM/48/}\HinterSignsSpace
\loneSign{\Aca GD/52/}\HinterSignsSpace
\loneSign{\Aca GE/61/}\HinterSignsSpace
\loneSign{\Aca GZ/34/}\HinterSignsSpace
\loneSign{\Aca GS/63/}\HinterSignsSpace
\loneSign{\Aca GN/69/}\HinterSignsSpace
\traittexte{5}\HinterSignsSpace
\Cadrat{\CadratLineI{\Aca GX/32/}\CadratLine{\Aca GU/62/}}\HinterSignsSpace
\loneSign{\Aca GG/32/}\HinterSignsSpace
\loneSign{\Aca GU/65/}\HinterSignsSpace
\loneSign{\Aca GM/48/}\HinterSignsSpace
\loneSign{\Aca GD/52/}\HinterSignsSpace
\loneSign{\Aca GO/59/}\HwordSpace
\Cadrat{\CadratLineI{\Aca GZ/33/}\CadratLine{\Aca GN/56/}}\HinterSignsSpace
\loneSign{\Aca GG/35/}\HinterSignsSpace
\loneSign{\Aca GZ/34/}\HinterSignsSpace
\Cadrat{\CadratLineI{\Aca GO/32/}\CadratLine{\Aca GD/52/}\CadratLine{\Aca GX/32/\hfill\Aca GD/88/}}\HinterSignsSpace
\loneSign{\Aca GO/35/}\HinterSignsSpace
\loneSign{\Aca GG/32/}\HinterSignsSpace
\loneSign{\Aca GG/32/}\HinterSignsSpace
\loneSign{\Aca GX/32/}\HinterSignsSpace
\Cadrat{\CadratLineI{\Aca GD/33/}\CadratLine{\Aca GZ/32/}}\HinterSignsSpace
\loneSign{\Aca GAa/43/}\HinterSignsSpace
\loneSign{\Aca GS/63/}\HinterSignsSpace
\loneSign{\Aca GZ/37/}\HinterSignsSpace
\Cadrat{\CadratLineI{\Aca GM/48/}\CadratLine{\Aca GN/66/}}\HinterSignsSpace
\loneSign{\Aca GA/43/}\HinterSignsSpace
\Cadrat{\CadratLineI{\Aca GZ/33/}\CadratLine{\Aca GN/66/}}\HinterSignsSpace
\Cadrat{\CadratLineI{\Aca GD/32/\hfill\Aca GZ/32/}\CadratLine{\Aca GD/69/}\CadratLine{\Aca GD/69/}}\HinterSignsSpace
\loneSign{\Aca GM/54/}\HinterSignsSpace
\Cadrat{\CadratLineI{\Aca GX/32/}\CadratLine{\Aca GN/66/}}\HinterSignsSpace
\loneSign{\Aca GZ/34/}\HinterSignsSpace
\Cadrat{\CadratLineI{\Aca GL/32/}\CadratLine{\Aca GD/52/}}\HinterSignsSpace
\loneSign{\Aca GG/77/}\HinterSignsSpace
\loneSign{\Aca GZ/34/}\HinterSignsSpace
\traittexte{6}\HinterSignsSpace
\Cadrat{\CadratLineI{\Aca GT/63/}\CadratLine{\Aca GD/52/}}\HinterSignsSpace
\Cadrat{\CadratLineI{\Aca GF/35/}\CadratLine{\Aca GX/32/\hfill\Aca GZ/32/}}\HwordSpace
\loneSign{\Aca GD/68/}\HinterSignsSpace
\Cadrat{\CadratLineI{\Aca GU/33/}\CadratLine{\Aca GD/35/}}\HinterSignsSpace
\loneSign{\Aca GG/32/}\HinterSignsSpace
\loneSign{\Aca GN/66/}\HinterSignsSpace
\loneSign{\Aca GS/63/}\HinterSignsSpace
\Cadrat{\CadratLineI{\Aca GD/35/}\CadratLine{\Aca GX/32/}}\HinterSignsSpace
\loneSign{\Aca GV/61/}\HinterSignsSpace
\Cadrat{\CadratLineI{\Aca GD/68/}\CadratLine{\Aca GA/47/}}\HinterSignsSpace
\Cadrat{\CadratLineI{\Aca GD/33/\hfill\Aca GZ/32/}\CadratLine{\Aca GN/66/}}\HinterSignsSpace
\Cadrat{\CadratLineI{\Aca GD/52/}\CadratLine{\Aca GV/44/}\CadratLine{\Aca GZ/33/}}\HinterSignsSpace
\Cadrat{\CadratLineI{\Aca GD/33/}\CadratLine{\Aca GD/52/}}\HinterSignsSpace
\loneSign{\Aca GS/63/}\HinterSignsSpace
\loneSign{\Aca GU/59/}\HinterSignsSpace
\loneSign{\Aca GG/63/}\HinterSignsSpace
\loneSign{\Aca GS/63/}\HinterSignsSpace
\loneSign{\Aca GN/66/}\HinterSignsSpace
\Cadrat{\CadratLineI{\Aca GU/70/}\CadratLine{\Aca GI/41/}}\HinterSignsSpace
\Cadrat{\CadratLineI{\ligAROBD}\negAROBvspace\negAROBvspace\negAROBvspace\negAROBvspace\negAROBvspace\CadratLine{{\Hsmaller\Hsmaller\Aca GS/63/}}}\HinterSignsSpace
\loneSign{\Aca GI/41/}\HwordSpace
\loneSign{\Aca GM/48/}\HinterSignsSpace
\loneSign{\Aca GS/63/}\HinterSignsSpace
\loneSign{\Aca GV/44/}\HinterSignsSpace
\Cadrat{\CadratLineI{\Aca GW/44/}\CadratLine{\Aca GD/52/}\CadratLine{\Aca GX/32/}}\HinterSignsSpace
\loneSign{\Aca GS/63/}\HinterSignsSpace
\Cadrat{\CadratLineI{\Aca GD/79/}\CadratLine{\Aca GAa/32/}}\HinterSignsSpace
\loneSign{\Aca GA/64/}\HinterSignsSpace
\Cadrat{\CadratLineI{\Aca GN/66/}\CadratLine{\Aca GI/41/}}\HinterSignsSpace
\loneSign{\Aca GS/63/}\HinterSignsSpace
\traittexte{7}\HinterSignsSpace
\Cadrat{\CadratLineI{\Aca GD/52/}\CadratLine{\Aca GAa/32/}}\HinterSignsSpace
\Cadrat{\CadratLineI{\Aca GN/66/}\CadratLine{\Aca GI/41/}}\HinterSignsSpace
\Cadrat{\CadratLineI{\Aca GD/86/}\CadratLine{\Aca GX/32/\hfill\Aca GX/32/}}\HinterSignsSpace
\Cadrat{\CadratLineI{\Aca GN/36/}\CadratLine{\Aca GZ/32/}}\HinterSignsSpace
\Cadrat{\CadratLineI{\Aca GQ/34/}\CadratLine{\Aca GN/66/}}\HinterSignsSpace
\Cadrat{\CadratLineI{\Aca GAa/32/}\CadratLine{\Aca GD/86/}\CadratLine{\Aca GX/32/}}\HinterSignsSpace
\Cadrat{\CadratLineI{\Aca GN/66/}\CadratLine{\Aca GI/41/}}\HinterSignsSpace
\loneSign{\Aca GV/59/}\HinterSignsSpace
\Cadrat{\CadratLineI{\Aca GM/33/}\CadratLine{\Aca GN/66/}}\HinterSignsSpace
\Cadrat{\CadratLineI{\Aca GX/32/}\CadratLine{\Aca GZ/37/}}\HinterSignsSpace
\Cadrat{\CadratLineI{\Aca GO/65/}\CadratLine{\Aca GQ/34/\hfill\Aca GO/81/}}\HinterSignsSpace
\Cadrat{\CadratLineI{\Aca GQ/34/}\CadratLine{\Aca GN/66/}}\HwordSpace
\loneSign{\Aca GN/66/}\HinterSignsSpace
\Cadrat{\CadratLineI{\Aca GU/38/}\CadratLine{\Aca GD/52/}}\HinterSignsSpace
\loneSign{\Aca GG/77/}\HinterSignsSpace
\Cadrat{\CadratLineI{\Aca GX/32/}\CadratLine{\Aca GAa/33/}}\HinterSignsSpace
\Cadrat{\CadratLineI{\Aca GU/33/}\CadratLine{\Aca GD/35/}}\HinterSignsSpace
\loneSign{\Aca GG/32/}\HinterSignsSpace
\loneSign{\Aca GG/32/}\HinterSignsSpace
\loneSign{\Aca GG/64/}\HinterSignsSpace
\loneSign{\Aca GI/41/}\HinterSignsSpace
\Cadrat{\CadratLineI{\Aca GD/52/}\CadratLine{\Aca GAa/32/\hfill\Aca GX/32/}}\HinterSignsSpace
\traittexte{8}\HinterSignsSpace
\Cadrat{\CadratLineI{\Aca GY/36/}\CadratLine{\Aca GN/66/}\CadratLine{\Aca GAa/32/}}\HinterSignsSpace
\Cadrat{\CadratLineI{\Aca GY/33/}\CadratLine{\Aca GZ/33/}}\HinterSignsSpace
\loneSign{\Aca GU/70/}\HinterSignsSpace
\Cadrat{\CadratLineI{\Aca GZ/32/}\CadratLine{\Aca GI/41/}}\HinterSignsSpace
\Cadrat{\CadratLineI{\Aca GD/35/}\CadratLine{\Aca GI/41/}}\HinterSignsSpace
\Cadrat{\CadratLineI{\Aca GU/33/}\CadratLine{\Aca GX/32/}\CadratLine{\Aca GY/33/}}\HinterSignsSpace
\Cadrat{\CadratLineI{\Aca GD/33/}\CadratLine{\Aca GZ/32/}}\HinterSignsSpace
\Cadrat{\CadratLineI{\Aca GN/56/}\CadratLine{\Aca GX/32/\hfill\Aca GZ/32/\hfill\Aca GZ/32/\hfill\Aca GZ/32/}\CadratLine{\Aca GI/41/}}\HinterSignsSpace
\loneSign{\Aca GN/66/}\HinterSignsSpace
\Cadrat{\CadratLineI{\Aca GG/73/}\CadratLine{\Aca GI/41/}}\HwordSpace
\cartouche{\loneSign{\Aca GN/36/}\HinterSignsSpace
\loneSign{\Aca GV/61/}\HinterSignsSpace
\Cadrat{\CadratLineI{\Aca GN/48/}\CadratLine{\Aca GN/48/}}}%
\HinterSignsSpace
\loneSign{\Aca GS/68/}\HinterSignsSpace
\Cadrat{\CadratLineI{\ligAROBDt}\CadratLine{\Aca GN/48/}}\HinterSignsSpace
\loneSign{\Aca GS/63/}\HinterSignsSpace
\loneSign{\Aca GF/52/}\HinterSignsSpace
\loneSign{\Aca GG/50/}\HinterSignsSpace
\loneSign{\Aca GS/63/}\HinterSignsSpace
\loneSign{\Aca GX/32/}\HinterSignsSpace
\Cadrat{\CadratLineI{\Aca GN/66/}\CadratLine{\InternalCadrat{\CadratLineI{\HquarterSpace }\CadratLine{\Aca GX/32/}}\hfill\Aca GG/77/}}\HinterSignsSpace
\loneSign{\Aca GZ/34/}\HinterSignsSpace
\loneSign{\Aca GG/50/}\HinterSignsSpace
\loneSign{\Aca GN/47/}\HinterSignsSpace
\Cadrat{\CadratLineI{\Aca GU/33/}\CadratLine{\Aca GD/52/}}\HinterSignsSpace
\loneSign{\Aca GM/48/}\HinterSignsSpace
\Cadrat{\CadratLineI{\Aca GN/69/}\CadratLine{\Aca GO/80/}}\HinterSignsSpace
\traittexte{9}\HinterSignsSpace
\loneSign{\Aca GG/56/}\HinterSignsSpace
\Cadrat{\CadratLineI{\Aca GX/32/}\CadratLine{\Aca GZ/33/}}\HinterSignsSpace
\Cadrat{\CadratLineI{\Aca GN/66/}\CadratLine{\Aca GX/32/\hfill\Aca GX/32/}}\HinterSignsSpace
\loneSign{\Aca GD/33/}\HinterSignsSpace
\Cadrat{\CadratLineI{\Hrevert{\Aca GI/38/}}\CadratLine{\Aca GX/32/}\CadratLine{\Aca GN/54/}}\HinterSignsSpace
\loneSign{\Aca GM/57/}\HinterSignsSpace
\loneSign{\Aca GG/77/}\HinterSignsSpace
\loneSign{\Aca GO/80/}\HwordSpace
\loneSign{\Aca GV/59/}\HinterSignsSpace
\Cadrat{\CadratLineI{\Aca GN/66/}\CadratLine{\Aca GD/69/}}\HinterSignsSpace
\loneSign{\Aca GN/47/}\HinterSignsSpace
\loneSign{\Aca GM/46/}\HinterSignsSpace
\Cadrat{\CadratLineI{\Aca GO/80/}\CadratLine{\Aca GX/32/\hfill\HquarterSpace \hfill\Aca GZ/32/}}\HinterSignsSpace
\loneSign{\Aca GV/60/}\HinterSignsSpace
\Cadrat{\CadratLineI{\Aca GS/63/}\CadratLine{\Aca GN/66/}}\HinterSignsSpace
\Cadrat{\CadratLineI{\Aca GD/32/}\CadratLine{\Aca GZ/33/}}\HinterSignsSpace
\Cadrat{\CadratLineI{\Aca GS/63/}\CadratLine{\Aca GN/66/}}\HinterSignsSpace
\loneSign{\Aca GG/50/}\HinterSignsSpace
\loneSign{\Aca GN/47/}\HinterSignsSpace
\Cadrat{\CadratLineI{\Aca GD/79/}\CadratLine{\Aca GV/35/}}\HinterSignsSpace
\loneSign{\Aca GG/32/}\HinterSignsSpace
\traittexte{10}\HinterSignsSpace
\Cadrat{\CadratLineI{\Aca GS/63/}\CadratLine{\Aca GN/66/}}\HinterSignsSpace
\loneSign{\Aca GF/66/}\HinterSignsSpace
\loneSign{\Aca GZ/34/}\HinterSignsSpace
\loneSign{\Aca GU/70/}\HinterSignsSpace
\Cadrat{\CadratLineI{\Aca GZ/32/}\CadratLine{\Aca GI/41/}}\HinterSignsSpace
\loneSign{\Aca GN/66/}\HinterSignsSpace
\loneSign{\Aca GG/54/}\HinterSignsSpace
\loneSign{\Aca GV/59/}\HinterSignsSpace
\loneSign{\Aca GV/59/}\HwordSpace
\loneSign{\Aca GV/59/}\HinterSignsSpace
\Cadrat{\CadratLineI{\Aca GN/66/}\CadratLine{\Aca GD/69/}}\HinterSignsSpace
\Cadrat{\CadratLineI{\ligAROBDt}\CadratLine{\Aca GN/48/}}}\end{hieroglyph}


\section{Table of signs}
\label{sec:signes}


\def\showHvalue#1#2{\hbox to 0.33\textwidth{#1\hfil
    #2\hfil}\linebreak[3]%
  \hskip 0.1mm  plus 1cm minus 1cm}
\EnPetit{\small
\noindent{}\input signlist.tex
}
\section{Thanks}
I want to thanks a few persons, whose help definitly helped improving
this package.  In 1993, M.~Louet and Corler, then students in computer
science at Brest University, wrote a X program for editing the fonts,
which proved very useful for touching a few signs that were really
ugly before. J\"urgen Kraus, egyptologist at Mainz university, has
provided me with numerous bug reports and suggestions. Helmar Wodtke
has sent me numerous ideas, and has published a number of Egyptology
books with hieroglyphic texts using both parts of Hiero\TeX{} and
software written by himself.

\bibliography{bib} 
\bibliographystyle{plain}

\end{document}

